\newpage
\section{Network Firewalls}

You are hired as the new network security advisor of Legitimate
Software Inc, a company that distributes decent software products. 

Currently, Legitimate Software products can be downloaded from the company
home page LSI.com, which is hosted on a HTTP web server within the company.
Programmers of Legitimate Software regularly update their products, and upload
the new software patches from company workstations to the web server.

\prob{20}
The company tech lead asks you to set up a firewall between company
machines (including the web server and workstations used by programmers) and
the external Internet, only allowing connections originating from a internal
host. 
\begin{enumerate}
\item List two different attacks (malicious activities) this firewall could
prevent. 
\item Describe a {\em legitimate} use of the company's
network that this firewall setup would also prevent.
\end{enumerate}

\sols{
  Attacks that could be prevented: Listing all hosts on the company's network (via ping); Logging onto the workstations using stolen passwords; Connecting to company databases from the Internet; 
  Legitimate uses that are also prevented: Download products from company's website from the Internet; Working from home (access workstations from the Internet, with correct credentials);
}
\eprob

\newpage
\prob{20} The tech lead asks you to fix the company's firewall configuration
to address the
shortcoming from
the previous question. Create a firewall
{\bf (A)~topology}; and {\bf (B) policies} that satisfy the following requirements:

\begin{itemize}
\item Customers
can visit the company's web site to download new software through HTTP (port
80), from anywhere on earth. 
\item Programmers in the company can upload files to
the web server through SSH (port 22), only when they are using the company
workstations. 
\item No network connections can be made to the company
workstations, unless the connection originate from one of the workstations. 
\item
Programmers can access the Internet (\eg, visit http://stackoverflow.com) on
their workstations. (Hint: divide the machines to multiple groups and set up
multiple firewalls in between.)
\end{itemize}
{\em Hint:} You will need more than one firewall. It will likely be faster to
write down your answer as a {\em picture} that shows the Internet, the company
web server, and the rest of the company network, as well as where you would
place firewalls and what policies should exist on each of them.

\sols{
  Internet <-> (||firewall1||) <-> web server <-> (||firewall2||) <-> workstations
  firewall1 minimum policies: drop connections from Internet, unless their destination is web server port 80
  firewall2 minimum policies: drop all connections from outside. drop connections from inside, except HTTP traffic to stackoverflow.com and SSH traffic to the web server
}
\eprob


\if 0
\prob{10} The network security of Legitimate Software is greatly enhanced by your
new firewall policy. However, a smart hacker managed to break into the
web server and replaced the software patches with his own malware. 

How would you prevent the illegitimate patches to be applied to Legitimate
products already installed on users' computers? You can assume that before the
attack, everything on users' computers are authentic, Legitimate products.
({\em Hint:} recall what you have learned about digital signatures and public
key infrastructures.)

\sols{
  All Legitimate products should only accept patches with digital signatures (certificates) created with company's private key (assume that this private key is not compromised). There's two possible authentication schemes: ship the software with company's public key, or register company's key with well know Certificate Authority.
}
\eprob
\fi