\documentclass[11pt]{article}

\usepackage{epsf}
\usepackage{epsfig}
\usepackage{url}
\usepackage{6829hw}
\input{macros.tex}

\begin{document}

\newcounter{listcount}
\newcounter{sublistcount}


\handout{P2}{October 1, 2011}{Instructor: Prof. Nick Feamster}
{College of Computing, Georgia Tech}{Final Project Assignment}


{\bf Due Date:} various dates for each “deliverable” (see below)

While the main goal of the case study project was to ``learn something'',
the focus of the final project is to “make / do something.” That is, your
project should result in an original artifact that relates to
information security — e.g., a piece of software, a novel discovery, a
demonstration, an original policy or design document, etc. Your group
should consist of 4 or 5 students. The scope of your project should be
significant and interesting enough to occupy you for about 4-5 weeks, and
should have a high likelihood of producing some tangible results in that
time span.

As with the case study, the ``deliverables'' for your final project are as follows:

\begin{itemize}
\itemsep=-1pt
\item {\bf Short proposal, due Tue 1 November (before class).} With your
  group, write a short proposal describing your plan for the final
  project, and start your initial work. Include the names all of your
  group’s members on the proposal, and have one group member upload it
  to the “Final Project” assignment in T-Square. See below for some
  potential kinds of projects. If you need additional suggestions or are
  unsure whether your ideas would be appropriate, don’t hesitate to ask
  the instructor.  (But try to do so sooner rather than later!)

\item {\bf Written report, due Thu 1 December.} Write a self-contained report
  briefly outlining the background and motivation for your project, and
  describing your findings and contributions (including any supporting
  artifacts). The appropriate length for the report will depend on the
  complexity of your project’s other components; for example, if you
  create a rich demonstration or piece of software, the report need not
  be as involved (but it should still summarize the key ideas and
  contributions).

\item {\bf Class presentation, last week and a half of classes.} Give a
  short presentation to the class summarizing your project and its
  outcomes. The presentation should cover only the most significant
  aspects of your project, and should be organized and well-rehearsed so
  that it fits within the allocated time (about 10 minutes, including
  questions).
\end{itemize}


Here is a (non-exhaustive) list of suggestions for possible types of projects:
\begin{itemize}
\itemsep=-1pt
\item Write an application, or an extension for an existing piece of
  software (e.g., Firefox, instant messengers, smartphone OSes, web
  applications) that enhances security in some way.
\item Study how a significant kind of malware operates, and devise a
  strategy for stopping its spread or mitigating its effects.
\item Write or extend a tool that automatically searches for common
  security flaws in a certain kind of application.
\item Learn and deploy a significant security tool in a system that you own or administer.
\item Formulate and/or demonstrate (within ethical boundaries) a
  significant new attack on the security of a system.
\item Propose a new design or security policy for an existing computing
  system — either a “clean slate,” or one that works within existing
  constraints.
\item Investigate and/or test what aspects of user interfaces make
  security-related functions more/less likely to be used correctly by
  humans, and the security implications.
\item Study and contribute to an existing open-source project or system
  that serves some security purpose (e.g., Tor, TrueCrypt, OpenID,
  password managers, software firewalls, etc.).
\item Survey/measure the relative real-world usage of different
  solutions to a security problem, and evaluate your findings
\end{itemize}

\end{document}
