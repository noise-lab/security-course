\documentclass[11pt]{article}

\usepackage{epsf}
\usepackage{epsfig}
\usepackage{url}
\usepackage{6829hw}

\newcommand{\newc}{\newcommand}

\newc{\code}[1]{{\tt #1}}
\newc{\func}[1]{{\em #1\/}}

\newc{\be}{\begin{enumerate}}
\newc{\ee}{\end{enumerate}}

\newc{\bi}{\begin{itemize}}
\newc{\ei}{\end{itemize}}

\newc{\bd}{\begin{description}}
\newc{\ed}{\end{description}}

\newc{\ov}[1]{$\overline{#1}$}
\newc{\instr}{\tt}

\newc{\doublespace}{\renewcommand{\baselinestretch}{1.5}}

\newcommand{\figref}[1]{Figure~\ref{#1}}
\newcommand{\tref}[1]{Table~\ref{#1}}

% Captioned table
\newc{\tbl}[3]{
        \begin{table}[htb]
                \centering
                #1
                \caption{#3}
                \label{#2}
        \end{table}
}

% Input a table.
\newcommand{\dblfig}[3]{
        \begin{figure}[htb]
		\centering
                \input{#1}
                \caption{#3}
                \label{#2}
        \end{figure}
}

\newcommand{\ddblfig}[4]{
        \begin{figure}[htb]
		\hspace{-0.1in}
                \psfig{figure=#1,width=0.45\textwidth}
                \caption{#3}
                \label{#2}
        \end{figure}
}

% Figure with no caption
\newcommand{\nofig}[2]{
        \begin{figure}[htb]
                \centering
                \psfig{figure=#1}
                \label{#2}
        \end{figure}
}

% Whole page figure
\newcommand{\schfig}[3]{
        \begin{figure}[p]
                \centering
                \psfig{figure=#1,height=7in}
                \caption{#3}
                \label{#2}
        \end{figure}
}

% Small figure
\newcommand{\sfig}[3]{
        \begin{figure}[ltb]
                \centering
               \hspace*{\fill}\rule{\linewidth}{.5mm}\hspace*{\fill}\vspace{3mm}
                \psfig{figure=#1,width=0.4\textwidth}
                \caption{#3}
                \label{#2}
               \vspace{3mm}\hspace*{\fill}\rule{\linewidth}{.5mm}\hspace*{\fill}
        \end{figure}
}

% Medium figure
\newcommand{\mfig}[3]{
        \begin{figure}[ltb]
		\centering
               \hspace*{\fill}\rule{\linewidth}{.5mm}\hspace*{\fill}\vspace{1mm}
                \psfig{figure=#1,height=2.5in}
                \caption{#3}
                \label{#2}
               \vspace{0mm}\hspace*{\fill}\rule{\linewidth}{.5mm}\hspace*{\fill}
        \end{figure}
}

\newcommand{\widefig}[4]{
        \begin{figure*}[htb]
                \centering
               \hspace*{\fill}\rule{\linewidth}{.5mm}\hspace*{\fill}\vspace{5mm}
                \psfig{figure=#1,width=#3}
                \caption{#4}
                \label{#2}
               \vspace{5mm}\hspace*{\fill}\rule{\linewidth}{.5mm}\hspace*{\fill}
        \end{figure*}
}

\newcommand{\mcfig}[4]{
        \begin{figure}[htbp]
                \centering
               \hspace*{\fill}\rule{\linewidth}{.5mm}\hspace*{\fill}\vspace{5mm}
                \psfig{figure=#1,width=#3}
                \caption{#4}
                \label{#2}
               \vspace{5mm}\hspace*{\fill}\rule{\linewidth}{.5mm}\hspace*{\fill}
        \end{figure}
}

\newcommand{\docfig}[3]{
        \begin{figure}[htbp]
               \hspace*{\fill}\rule{\linewidth}{.5mm}\hspace*{\fill}\vspace{5mm}
                \centering
                \psfig{figure=#1,width=#3}
                \label{#2}
               \vspace{5mm}\hspace*{\fill}\rule{\linewidth}{.5mm}\hspace*{\fill}
        \end{figure}
}

% Medium-large figure
\newcommand{\mlfig}[3]{
        \begin{figure}[htb]
                \centering
               \hspace*{\fill}\rule{\linewidth}{.5mm}\hspace*{\fill}\vspace{5mm}
                \psfig{figure=#1,height=3.25in}
                \caption{#3}
                \label{#2}
               \vspace{5mm}\hspace*{\fill}\rule{\linewidth}{.5mm}\hspace*{\fill}
        \end{figure}
}

% Large figure
\newcommand{\lfig}[3]{
        \begin{figure}[p]
                \centering
               \hspace*{\fill}\rule{\linewidth}{.5mm}\hspace*{\fill}\vspace{5mm}
                \psfig{figure=#1,height=5in}
                \caption{#3}
                \label{#2}
               \vspace{5mm}\hspace*{\fill}\rule{\linewidth}{.5mm}\hspace*{\fill}
        \end{figure}
}

% 'gg' figures are the double column versions of the 'g' figures above.
\newcommand{\sfigg}[3]{
        \begin{figure*}[htb]
                \centering
               \hspace*{\fill}\rule{\linewidth}{.5mm}\hspace*{\fill}\vspace{5mm}
                \psfig{figure=#1,height=1.5in}
                \caption{#3}
                \label{#2}
               \vspace{5mm}\hspace*{\fill}\rule{\linewidth}{.5mm}\hspace*{\fill}
        \end{figure*}
}

% Medium figure
\newcommand{\mfigg}[3]{
        \begin{figure*}
                \centering
               \hspace*{\fill}\rule{\linewidth}{.5mm}\hspace*{\fill}\vspace{5mm}
                \psfig{figure=#1,width=\linewidth}
                \caption{#3}
                \label{#2}
               \vspace{0mm}\hspace*{\fill}\rule{\linewidth}{.5mm}\hspace*{\fill}
        \end{figure*}
}

% Medium-large figure
\newcommand{\mlfigg}[3]{
        \begin{figure*}[htb]
                \centering
               \hspace*{\fill}\rule{\linewidth}{.5mm}\hspace*{\fill}\vspace{5mm}
                \psfig{figure=#1,height=3.25in}
                \caption{#3}
                \label{#2}
               \vspace{5mm}\hspace*{\fill}\rule{\linewidth}{.5mm}\hspace*{\fill}
        \end{figure*}
}

% Large figure
\newcommand{\lfigg}[3]{
        \begin{figure*}[p]
                \centering
               \hspace*{\fill}\rule{\linewidth}{.5mm}\hspace*{\fill}\vspace{5mm}
                \psfig{figure=#1,height=5in}
                \caption{#3}
                \label{#2}
               \vspace{5mm}\hspace*{\fill}\rule{\linewidth}{.5mm}\hspace*{\fill}
        \end{figure*}
}

% Variable size figure
\newcommand{\vfigg}[4]{
        \begin{figure*}[htb]
                \centering
               \hspace*{\fill}\rule{\linewidth}{.5mm}\hspace*{\fill}\vspace{5mm}
                \psfig{figure=#1,#2}
                \caption{#4}
                \label{#3}
               \vspace{5mm}\hspace*{\fill}\rule{\linewidth}{.5mm}\hspace*{\fill}
        \end{figure*}
}

\newcommand{\vfig}[4]{
        \begin{figure}[ltb]
                \centering
               \hspace*{\fill}\rule{\linewidth}{.5mm}\hspace*{\fill}\vspace{1mm}
                \psfig{figure=#1,#2}
                \caption{#4}
                \label{#3}
               \vspace{1mm}\hspace*{\fill}\rule{\linewidth}{.5mm}\hspace*{\fill}
        \end{figure}
}

\newcommand{\vnlfig}[4]{
        \begin{figure}[htb]
                \centering
               \hspace*{\fill}\rule{\linewidth}{0mm}\hspace*{\fill}\vspace{5mm}
                \psfig{figure=#1,#2}
                \caption{#4}
                \label{#3}
               \vspace{0mm}\hspace*{\fill}\rule{\linewidth}{0mm}\hspace*{\fill}
        \end{figure}
}

\newcommand{\dblvfig}[6]{
        \begin{figure}[htb]
                \centering
                \hspace*{\fill}\rule{\linewidth}{0mm}\hspace*{\fill}\vspace{0.5mm}
                \psfig{figure=#1,#2}
	        \hspace{1in}
                \psfig{figure=#3,#4}
                \caption{#6}
                \label{#5}
               \vspace{2mm}\hspace*{\fill}\rule{\linewidth}{0mm}\hspace*{\fill}
        \end{figure}
}
\newc{\myspacing}{
        \let\oldtextheight=\textheight
        \let\oldtextwidth=\textwidth

        \let\oldtopmargin=\topmargin
        \let\oldheadheight=\headheight
        \let\oldfootheight=\footheight
        \let\oldheadsep=\headsep
        \let\oldoddsidemargin=\oddsidemargin


        \textheight 8.5in
        \textwidth 6in

        \topmargin 0in
        \headheight 0in
        \footheight 1.5in
        \headsep 0in
        \oddsidemargin 0in

}

\newc{\oldspacing}{
        \let\textheight=\oldtextheight 
        \let\textwidth=\oldtextwidth

        \let\topmargin=\oldtopmargin 
        \let\headheight=\oldheadheight 
        \let\footheight=\oldfootheight
        \let\headsep=\oldheadsep
        \let\oddsidemargin=\oldoddsidemargin
}
% Local Variables: 
% mode: latex
% TeX-master: t
% End: 


\begin{document}

\newcounter{listcount}
\newcounter{sublistcount}


\handout{PS3}{September 15, 2011}{Instructor: Prof. Nick Feamster}
{College of Computing, Georgia Tech}{Problem Set 3: Vulnerabilities}

This problem set has three questions, each with several parts.  Answer
them as clearly and concisely as possible.  You may discuss ideas with
others in the class, but your solutions and presentation must be your
own.  Do not look at anyone else's solutions or copy them from
anywhere. (Please refer to the Georgia Tech honor code, posted on the
course Web site).

Turn in your writeup {\bf September 29, 2011} by 11:59pm.
{\em Please upload your solutions to T-Square.  Other forms of
  submission will not be accepted!}  We will be providing more
information about how to turn in your assignment as the due date
approaches.

\begin{enumerate}
\item {\bf 15 points} Pfleeger and Pfleeger, Section 3.10, Exercise
  15. For the ``design requirements'', concentrate on specifying a
  security policy for the processor, in terms of both its abstract
  functionality and physical properties.

\item {\bf 15 points} Consider Automated Teller Machines (ATMs), which
  use a customer’s bank card and secret Personal Identification Number
  (PIN) for common banking tasks like withdrawals, checking account
  balances, etc.

Identify at least three distinct input/output paths on an ATM, and name the
endpoints of each.

For each of the above paths, describe the extent to which it qualifies
(or doesn't) as a ``trusted I/O path''. Focus on confidentiality and
integrity of data as it travels between the two endpoints. For this
analysis, you may want to research some of the relevant known attacks on
ATMs.

\item {\bf 20 points} Find an example of a specific security flaw in a
  commonly used commercial operating system (e.g., Windows, Linux, Mac
  OS X) in the past year. Good sources include CERT, the Microsoft
  Security TechCenter, or {\tt securityfocus.com}.

  Choose a flaw that has very significant security implications (e.g.,
  favor “arbitrary remote code execution” over “local denial of
  service”).  Give a high-level summary of the flaw and its
  implications, in your own words.  Classify the nature and cause of
  the vulnerability: for example, is it the result of flawed code in an
  unsafe language? an incomplete or inconsistent specification? a flawed
  design in terms of modularity and/or encapsulation? Justify your
  answer.

\item {\bf 20 points} In UNIX, the Internet Daemon (now called xinetd on
  some versions of UNIX) provides the handshaking that occurs when a
  TCP/IP connection.  Xinetd is susceptible to a Denial of Serice attack,
  where many connections are made to the same service.  When too many
  connections are made within a specified short period of time, xinetd
  will terminate that service for a short period of time and print error
  messages of the form:

\begin{verbatim}
inetd[354]: telnet/fcp server failing (looping), service terminated
\end{verbatim}

Various attack programs exist to launch thousands of connections on a
specific port, overloading the machine.  See
\url{http://www.cotse.com/dos.htm} for some examples of source code
designed to mount denial of service attacks.  

Explain the design alternatives that the designers of an Internet
service like xinetd considered when they decided to implement inetd.
What are some alternative designs that solve this vulnerability?  Do
they introduce new vulnerabilities?


\item  {\bf 30 points} In this problem, you will try to understand how UNIX generates
  password files, and then try to crack some passwords!
\begin{itemize}
\item Examine the source code or man page for crypt.  How does this
  program take a plaintext password and generate the ciphertext that we
  see in /etc/password, or /etc/shadow? What cipher is used to generate
  the cipher from the plaintext?

\item Using crypt(3) on a UNIX machine, generate the ciphertext for security
and netsecurity. What do you observe? Why? 

\item {\tt passwd} typically uses something called a “salt” to generate
  the password for each user. Why? 

\item
Consider the following password file generated with crypt(3):
\begin{verbatim}
root:IWpIzqD0jR1.c:100:100:Charlie Root:/home/root:/bin/sh 
cs4251:UNzrFi5aYL9DU:101:101:CS4251:/home/cs4251:/bin/sh 
mysql:WqCBVG36lcuAc:102:102:MySQL:/home/mysql:/bin/sh 
guest:FTQinpjr.VRM.:103:103:Guest:/home/guest:/bin/sh 
test:LF2c9qM5l6X7Q:104:104:Testing:/home/test:/bin/sh
\end{verbatim}

Run the default mode of John the Ripper
(\url{http:www.openwall.com/john/}) on the password file.  One of the
passwords will be cracked.  Which one?  Why (which rule of John was
applied)?  One of these passwords will be cracked. Which one? Why (which
rule of John was applied)?

\item Try the ``wordlist'' mode of John.  Which password is cracked now?
  Which rule of John was applied?  ({\em Hint:} John's default wordlist
  is very small by default.  You may have to augment this wordlist with
  one of your own.)

\item One of the users has a password that is a rotated version of a
  dictionary word. Modify John’s rule list to incorporate this
  feature. Which password does this now reveal? Please include the
  source for your modifications to the rule list.

\item One user is predisposed to using leetspeak (4 for a/A, 1 for i/I
  and l, 3 for e/E). His password is also a dictionary word. Modify
  john.conf to incorporate this feature. Which password is revealed?

\item One user likes to swap two adjacent characters of a dictionary
  word. Can you modify john.conf to do this using existing syntax? If
  not, how can you incorporate this feature? What is the password that
  is revealed?

\end{itemize}

\end{enumerate}



\end{document}
