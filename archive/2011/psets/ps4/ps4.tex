\documentclass[11pt]{article}

\usepackage{epsf}
\usepackage{epsfig}
\usepackage{url}
\usepackage{6829hw}

\newcommand{\newc}{\newcommand}

\newc{\code}[1]{{\tt #1}}
\newc{\func}[1]{{\em #1\/}}

\newc{\be}{\begin{enumerate}}
\newc{\ee}{\end{enumerate}}

\newc{\bi}{\begin{itemize}}
\newc{\ei}{\end{itemize}}

\newc{\bd}{\begin{description}}
\newc{\ed}{\end{description}}

\newc{\ov}[1]{$\overline{#1}$}
\newc{\instr}{\tt}

\newc{\doublespace}{\renewcommand{\baselinestretch}{1.5}}

\newcommand{\figref}[1]{Figure~\ref{#1}}
\newcommand{\tref}[1]{Table~\ref{#1}}

% Captioned table
\newc{\tbl}[3]{
        \begin{table}[htb]
                \centering
                #1
                \caption{#3}
                \label{#2}
        \end{table}
}

% Input a table.
\newcommand{\dblfig}[3]{
        \begin{figure}[htb]
		\centering
                \input{#1}
                \caption{#3}
                \label{#2}
        \end{figure}
}

\newcommand{\ddblfig}[4]{
        \begin{figure}[htb]
		\hspace{-0.1in}
                \psfig{figure=#1,width=0.45\textwidth}
                \caption{#3}
                \label{#2}
        \end{figure}
}

% Figure with no caption
\newcommand{\nofig}[2]{
        \begin{figure}[htb]
                \centering
                \psfig{figure=#1}
                \label{#2}
        \end{figure}
}

% Whole page figure
\newcommand{\schfig}[3]{
        \begin{figure}[p]
                \centering
                \psfig{figure=#1,height=7in}
                \caption{#3}
                \label{#2}
        \end{figure}
}

% Small figure
\newcommand{\sfig}[3]{
        \begin{figure}[ltb]
                \centering
               \hspace*{\fill}\rule{\linewidth}{.5mm}\hspace*{\fill}\vspace{3mm}
                \psfig{figure=#1,width=0.4\textwidth}
                \caption{#3}
                \label{#2}
               \vspace{3mm}\hspace*{\fill}\rule{\linewidth}{.5mm}\hspace*{\fill}
        \end{figure}
}

% Medium figure
\newcommand{\mfig}[3]{
        \begin{figure}[ltb]
		\centering
               \hspace*{\fill}\rule{\linewidth}{.5mm}\hspace*{\fill}\vspace{1mm}
                \psfig{figure=#1,height=2.5in}
                \caption{#3}
                \label{#2}
               \vspace{0mm}\hspace*{\fill}\rule{\linewidth}{.5mm}\hspace*{\fill}
        \end{figure}
}

\newcommand{\widefig}[4]{
        \begin{figure*}[htb]
                \centering
               \hspace*{\fill}\rule{\linewidth}{.5mm}\hspace*{\fill}\vspace{5mm}
                \psfig{figure=#1,width=#3}
                \caption{#4}
                \label{#2}
               \vspace{5mm}\hspace*{\fill}\rule{\linewidth}{.5mm}\hspace*{\fill}
        \end{figure*}
}

\newcommand{\mcfig}[4]{
        \begin{figure}[htbp]
                \centering
               \hspace*{\fill}\rule{\linewidth}{.5mm}\hspace*{\fill}\vspace{5mm}
                \psfig{figure=#1,width=#3}
                \caption{#4}
                \label{#2}
               \vspace{5mm}\hspace*{\fill}\rule{\linewidth}{.5mm}\hspace*{\fill}
        \end{figure}
}

\newcommand{\docfig}[3]{
        \begin{figure}[htbp]
               \hspace*{\fill}\rule{\linewidth}{.5mm}\hspace*{\fill}\vspace{5mm}
                \centering
                \psfig{figure=#1,width=#3}
                \label{#2}
               \vspace{5mm}\hspace*{\fill}\rule{\linewidth}{.5mm}\hspace*{\fill}
        \end{figure}
}

% Medium-large figure
\newcommand{\mlfig}[3]{
        \begin{figure}[htb]
                \centering
               \hspace*{\fill}\rule{\linewidth}{.5mm}\hspace*{\fill}\vspace{5mm}
                \psfig{figure=#1,height=3.25in}
                \caption{#3}
                \label{#2}
               \vspace{5mm}\hspace*{\fill}\rule{\linewidth}{.5mm}\hspace*{\fill}
        \end{figure}
}

% Large figure
\newcommand{\lfig}[3]{
        \begin{figure}[p]
                \centering
               \hspace*{\fill}\rule{\linewidth}{.5mm}\hspace*{\fill}\vspace{5mm}
                \psfig{figure=#1,height=5in}
                \caption{#3}
                \label{#2}
               \vspace{5mm}\hspace*{\fill}\rule{\linewidth}{.5mm}\hspace*{\fill}
        \end{figure}
}

% 'gg' figures are the double column versions of the 'g' figures above.
\newcommand{\sfigg}[3]{
        \begin{figure*}[htb]
                \centering
               \hspace*{\fill}\rule{\linewidth}{.5mm}\hspace*{\fill}\vspace{5mm}
                \psfig{figure=#1,height=1.5in}
                \caption{#3}
                \label{#2}
               \vspace{5mm}\hspace*{\fill}\rule{\linewidth}{.5mm}\hspace*{\fill}
        \end{figure*}
}

% Medium figure
\newcommand{\mfigg}[3]{
        \begin{figure*}
                \centering
               \hspace*{\fill}\rule{\linewidth}{.5mm}\hspace*{\fill}\vspace{5mm}
                \psfig{figure=#1,width=\linewidth}
                \caption{#3}
                \label{#2}
               \vspace{0mm}\hspace*{\fill}\rule{\linewidth}{.5mm}\hspace*{\fill}
        \end{figure*}
}

% Medium-large figure
\newcommand{\mlfigg}[3]{
        \begin{figure*}[htb]
                \centering
               \hspace*{\fill}\rule{\linewidth}{.5mm}\hspace*{\fill}\vspace{5mm}
                \psfig{figure=#1,height=3.25in}
                \caption{#3}
                \label{#2}
               \vspace{5mm}\hspace*{\fill}\rule{\linewidth}{.5mm}\hspace*{\fill}
        \end{figure*}
}

% Large figure
\newcommand{\lfigg}[3]{
        \begin{figure*}[p]
                \centering
               \hspace*{\fill}\rule{\linewidth}{.5mm}\hspace*{\fill}\vspace{5mm}
                \psfig{figure=#1,height=5in}
                \caption{#3}
                \label{#2}
               \vspace{5mm}\hspace*{\fill}\rule{\linewidth}{.5mm}\hspace*{\fill}
        \end{figure*}
}

% Variable size figure
\newcommand{\vfigg}[4]{
        \begin{figure*}[htb]
                \centering
               \hspace*{\fill}\rule{\linewidth}{.5mm}\hspace*{\fill}\vspace{5mm}
                \psfig{figure=#1,#2}
                \caption{#4}
                \label{#3}
               \vspace{5mm}\hspace*{\fill}\rule{\linewidth}{.5mm}\hspace*{\fill}
        \end{figure*}
}

\newcommand{\vfig}[4]{
        \begin{figure}[ltb]
                \centering
               \hspace*{\fill}\rule{\linewidth}{.5mm}\hspace*{\fill}\vspace{1mm}
                \psfig{figure=#1,#2}
                \caption{#4}
                \label{#3}
               \vspace{1mm}\hspace*{\fill}\rule{\linewidth}{.5mm}\hspace*{\fill}
        \end{figure}
}

\newcommand{\vnlfig}[4]{
        \begin{figure}[htb]
                \centering
               \hspace*{\fill}\rule{\linewidth}{0mm}\hspace*{\fill}\vspace{5mm}
                \psfig{figure=#1,#2}
                \caption{#4}
                \label{#3}
               \vspace{0mm}\hspace*{\fill}\rule{\linewidth}{0mm}\hspace*{\fill}
        \end{figure}
}

\newcommand{\dblvfig}[6]{
        \begin{figure}[htb]
                \centering
                \hspace*{\fill}\rule{\linewidth}{0mm}\hspace*{\fill}\vspace{0.5mm}
                \psfig{figure=#1,#2}
	        \hspace{1in}
                \psfig{figure=#3,#4}
                \caption{#6}
                \label{#5}
               \vspace{2mm}\hspace*{\fill}\rule{\linewidth}{0mm}\hspace*{\fill}
        \end{figure}
}
\newc{\myspacing}{
        \let\oldtextheight=\textheight
        \let\oldtextwidth=\textwidth

        \let\oldtopmargin=\topmargin
        \let\oldheadheight=\headheight
        \let\oldfootheight=\footheight
        \let\oldheadsep=\headsep
        \let\oldoddsidemargin=\oddsidemargin


        \textheight 8.5in
        \textwidth 6in

        \topmargin 0in
        \headheight 0in
        \footheight 1.5in
        \headsep 0in
        \oddsidemargin 0in

}

\newc{\oldspacing}{
        \let\textheight=\oldtextheight 
        \let\textwidth=\oldtextwidth

        \let\topmargin=\oldtopmargin 
        \let\headheight=\oldheadheight 
        \let\footheight=\oldfootheight
        \let\headsep=\oldheadsep
        \let\oddsidemargin=\oldoddsidemargin
}
% Local Variables: 
% mode: latex
% TeX-master: t
% End: 


\begin{document}

\newcounter{listcount}
\newcounter{sublistcount}


\handout{PS4}{November 8, 2011}{Instructor: Prof. Nick Feamster}
{College of Computing, Georgia Tech}{Problem Set 4: Attacks}

This problem set has three questions, each with several parts (plus a
fourth fun activity).  Answer them as clearly and concisely as possible.
You may discuss ideas with others in the class, but your solutions and
presentation must be your own.  Do not look at anyone else's solutions
or copy them from anywhere. (Please refer to the Georgia Tech honor
code, posted on the course Web site).

Turn in your writeup {\bf November 22, 2011} by 11:59pm.
{\em Please upload your solutions to T-Square.  Other forms of
  submission will not be accepted!}  We will be providing more
information about how to turn in your assignment as the due date
approaches.

\begin{enumerate}
\item {\bf One-Time Pads (30 points).} Eve has been eavesdropping on
  Alice and Bob's communications with each other for some time. They
  appear to be using a one-time pad to keep their messages secret. Eve
  suspects that the plaintexts are just English sentences encoded in the
  standard ASCII character set, and the ciphertexts are generated using
  bitwise exclusive-or (XOR) with the pad. For example, in ASCII the
  character 'a' has hexadecimal value 61 (or 01100001 in binary), which
  when bitwise-XOR’ed with the hexadecimal pad value 83 (10000011 in
  binary) yields the hexadecimal ciphertext e2 (11100010 in binary).

Knowing that the one-time pad is hard to use properly, Eve has been
storing every ciphertext sent between Alice and Bob, and XORing pairs of
them to look for any anomalies. One day she notices that a pair of
ciphertexts XOR to a value (shown below in hexademical) that appears
``strange''. She suspects that Alice and Bob may have reused part of their
pad, and asks you to recover the plaintexts.

\begin{itemize}
\itemsep=-1pt
\item (10 points.) Why has Eve been XORing pairs of ciphertexts? What is
  ``strange'' about the XOR value below that she found?
\item (10 points.) Formulate and describe your approach for helping
  Eve. The messages may be time-sensitive, so your attack should work as
  quickly as possible. 
\item (10 points.) Give as much of the plaintexts as you can find.
\end{itemize}

\begin{verbatim}
03 03 0b 4f 45 5b 48 09 0b 54 54 1b 4f 1d 0d 12 45 57 0c 54 48 00
02 45 4e 2a 19 0b 09 53 00 3a 55 1f 19 15 01 07 45 48 11 17 17 54
0b 5a 55 53 28 05 4b 0a 55 01 55 02 04 44 58 4f 42 00 07 45 49 1b
52 01 00 1f 1c 0a 4f 15 0b 01 1c 00 1e 0e 44 42 1a 08 00 17 0d 04
4c 44 42 48 53 2b 51 11 00 11 06 00 43 54 4f 10 02 45 13 42 01 1a
00 49 0a 11 00
\end{verbatim}

\newpage
\item {\bf Port Scanning (40 points).}  In this hands-on problem, you will
  re-create some of the results that we performed in lecture, using the
  {\tt nmap}.  
\begin{itemize}
\item (10 points) Explain how port scanning works, and why an attacker
  might run a port scanning tool.
\item Download the nmap tool and install it somewhere where you can run
it.  (\url{http://nmap.org/download.html}).
\item (5 points) Run {\tt nmap} against {\tt
  porter-square.cc.gt.atl.ga.us}.  What ports do you see open on the
  machine?  How did {\tt nmap} discover this? Copy the output of running
  the tool into your writeup.  Based on the list of open ports, make
  your best guess at the services running on this machine. {\em Extra
    credit:} Extra credit if you can figure out the versions of services
  running on the machine! (This will require tools other than {\tt
    nmap}.)
\item (5 points) Use {\tt nmap} to determine the version of the operating system
  that {\tt porter-square} is running.  What operating system is
  running?  How does {\tt nmap} determine the operating system?
\item (10 points) Use {\tt nmap} (or, if you prefer, a script) to determine
  all hosts on {\tt 130.207.0.0/16} that are running a Web server.  {\em
    Hint:} You can use some options in {\tt nmap} to scan an entire
  network subnet, and to restrict the scanning to a particular port.
  This will go much, much faster if you ise the ``-p'' option in {\tt
    nmap}. 
\item (10 points) Find a machine on the wide-area Internet that has
  the Simple Mail Transport Protocol (SMTP, port 25) open.  List the IP
  address of the machine that you found and show the output from your
  {\tt nmap} tool. This is called an ``open mail relay''. How might an
  attacker be able to use an open mail relay?
\end{itemize}

\item {\bf Search Engine Optimization Competition (30 points + quiz
  bonus points).}  {\em Complete in your project groups!} Construct a
  Web page that comes up \#1 (or as high as you can manage) in Google's
  search engine when a user searches ``radiator palace summit seaweed''.
  The winner of the competition will be judged {\em in class} on
  November 22.  All members of the group will receive 5 quiz bonus
  points!

\item {\bf For fun: Tor.}  Run and install Tor
  (\url{http://torproject.org/}, if you haven't done so before. More to
  come on anonymity and privacy on the next (and last!) problem set!

\end{enumerate}



\end{document}
