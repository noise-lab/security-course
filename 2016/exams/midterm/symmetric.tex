\newpage
\section{Symmetric Key Cryptography}

\prob{30}
Answer the following questions regarding a communication protocol.

Alice and Bob share two secrets, $k_1$ and $k_2$, which are initially
unknown to any third party. Alice wants to send a message $m$ to Bob;
she transmits the follow content:
$$c = E_{k_1} (m || H(k_2 || m) )$$

where $E_{k_1}$ is the symmetric encryption function using key $k_1$,
$H$ is a cryptographically secure hash function (\eg, SHA-1), and $||$ is
the concatenation operator.

\begin{enumerate}
    \item When Bob receives $c$, how does he compute and verify the original message $m$? Please list all the necessary steps.

\sols{
    \textit{Answer:}
    \begin{enumerate}
        \item Compute $De_{k_1}(c) = m || hash$, where $De$ is the decryption function that corresponds to the encryption function $En$. Note: $En$ and $De$ should actually be the same function for a stream cipher.
        \item Concatenate $k_2$ and $m$ to compute $H(k_2 || m) = hash'$
        \item Compare $hash$ and $hash'$, if they are the same then Bob can successfully retrieve the message $m$. Otherwise there might be an error during transmission.
    \end{enumerate}
}

     \item Does this protocol provide the following properties:
       \begin{enumerate}
         \item Confidentiality (such that no third party may retrieve $m$ from the transmitted content $c$)?
            \sols{\textit{Answer:} Yes. The message is encrypted by $k_1$, which
            is unknown to any third party.}
         \item Integrity (such that Bob is aware of any alterations to the transmitted content $c$)?
            \sols{\textit{Answer:} Yes. As shown in part 1, any alterations can
            be detected by the mismatch of transmitted $hash$ and computed $hash'$.}
         \item Non-repudiation (such that \underline{Alice} cannot deny previously sent messages)?
            \sols{\textit{Answer:} No. Since both Alice and Bob know $k_1, k_2$,
            Bob can forge a message and claim that it was sent by Alice. Similarly, Alice can create a valid message and claim that it was created by Bob.}
         \end{enumerate}

    \item In the event that the protocol above does not provide
      confidentiality, integrity or 
      non-repudiation, provide a fix to the protocol to make
      it more secure. 

\sols{
        \textit{Possible answers:} 
        \begin{enumerate}
            \item PKI: Alice encrypts with her private key and Bob
              decrypts with the public key. Bob receives Alice's public
              key from trusted CA. 
            \item Alice add a sequence number in the transmitted
              message, she also publishes the most recent used sequence
              number to a trusted third party. In this case Alice cannot
              deny any message she has sent, while it is easy to disavow
              a message forged by a malicious third party (sequence
            number greater than maximum). 
        \end{enumerate}
}

\end{enumerate}
\eprob

\newpage
\prob{10} Does Diffie-Hellman key exchange ensure that two
parties can always exchange messages securely? If so, explain why. If
not, describe a possible attack. \eprob
\vspace*{0.5in}

\sols{\textit{Answer:} No. MITM attack is possible. With DH you don’t know who’s on the other end, could be Mallory.}

\prob{20} Alice now wants to send encrypted messages to Bob using AES.
The two parties do not share any secret key, so Alice first needs to 
share a AES key $k$ with Bob. We assume that the AES key is 128 bits. Alice
knows Bob's 4096-bit RSA public key is (3, $N$), so she encrypted the key and
sent $c_k = k^3$ mod $N$ to Bob. 

\begin{enumerate} 
\item Can an eavesdropper who intercepts this message learn
the AES key? Explain why or why not.  

\sols{\textit{Answer:}Since Bob's public exponent is very small and the AES key is only 128 bit long, $c_k$ will always be less than $N$ and hence the eavesdropper can easily recover $k$ by taking the cube root of $c_k$.}

\item If a vulnerability exists, propose a way to fix the problem.

\sols{\textit{Answer:} The best fix is to add padding and then do OAEP encoding. Another possible but not ideal fix could be to increase Bob's public exponent size and perhaps lower the size of $N$ This solution requires changing the public key, thus it is not very practical.}

\end{enumerate}
\eprob
\vspace*{1.5in}

