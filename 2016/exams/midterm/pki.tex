\newpage
\section{Public Key Infrastructure}

You've recently purchased a Mest smart webcam for your
dorm room that decides to communicate with a cloud-hosted server at {\tt
  mest.com}. The webcam also has a client certificate that is known
to the {\tt mest.com} server. 

Suppose that the webcam periodically takes a photograph, then encrypts
and signs the photograph before uploading it to the Mest cloud server.

\prob{30}
In a rush to production, Mest ships the webcam with the list of
the roughly 200 trusted root certificate authorities (CAs) that are in the
Firefox browser.

\begin{enumerate}
\item Explain why shipping the webcam with a list of so many root
  CAs might not be necessary in the case of a device like a webcam.

\sols{{\em Answer:} Because the device is likely to need to speak to only a few
Internet destinations (as opposed to arbitrary websites), it is reasonable to assume
that those small number of destinations could be signed by a single CA that is trusted
by the manufacturer.}

\item One of the root CAs happens to be from an organization,
  In-The-Clear Rootin' (ITCR), which happens to be controlled by a
  foreign nation-state actor.  Can ITCR mount an attack on
  confidentiality, integrity, or both?  If so, describe the attack. If
  not, explain why not.

\sols{{\em Answer:} ITCR can mount an attack on both confidentiality and integrity
via a man-in-the-middle (MITM) attack. Specifically, ITCR can sign a certificate
for the domain of the manufacturer, and the client would trust the signatures corresponding
to this public/private key pair because it was signed by a root CA.}

\item Mest decides to revoke ITCR's root CA from all of its deployed
  webcams with a signed, secure software update to the webcam's
  firmware, which it sends to each camera over the Internet. IS the
  software update itself vulnerable to an attack from ITCR? Why or why
  not? 

\sols{{\em Answer:} The software update itself is vulnerable to attack, just as
any other message would be. There's no guarantee that the integrity of the message
itself is preserved, due to the possibilty of a MITM attack on the update itself.}

\end{enumerate}
\eprob

\newpage
\prob{20} Your dorm room was robbed, but fortunately, you've caught the
perpetrator on one of the photos from your Mest camera, and the camera
uploaded the photo to {\tt mest.com}! \footnote{Suppose the
photo also includes necessary metadata, like a timestamp.}.

You log into the {\tt mest.com} the photo to the campus police.  

Before going after the perpetrator, though, they want to make sure that your
photo is authentic, and actually came from the camera in your dorm
room. 

\begin{enumerate}
\item When you visit {\tt mest.com} on your browser, you notice that
  {\tt mest.com} is using a self-signed 
  certificate for its website. In the certificate, you notice that the
  name of the organization is ``Mest, Inc.''. Can you be certain that
  you're visiting the website of the organization that manufactured
  your webcam?  Why or why not?

\sols{{\em Answer:} You cannot be certain. A self-signed certificate is simply signed
by the organization of the site that you are trying to visit (in this case, Mest).
But, any attacker could simply generate a rogue certificate for Mest, sign it themselves,
and present that to a client. That scenario is indistinguishable from a legitimate
Mest site/certificate.}

\item Explain how the camera's {\em client certificate}---and
  an associated protocol---could link the photo to the camera in your
  dorm room. Be sure to list {\em everything that the certificate would
    need to contain} to help the police verify that the photo came from
  your camera.

\sols{{em Answer:} The basic idea here would be to have a certificate that is associated
with the client's hardware (\eg, serial number), that is signed by the manufacturer
and thus could later be checked and tied directly to the camera. We accepted many
variations of this idea here.}

\end{enumerate}
\eprob

