\section{Ethics}

On Friday, October 21, 2016, a major denial of service attack was
mounted using Internet of Things (IoT) devices which were infected by
malware based on their reliance of default passwords. The botnet that
infected these vulnerable devices was called the ``Mirai'' botnet, and
mounted denial of service attacks against Twitter, Github, Reddit, and
many other Internet sites.

Troubled by this turn of events, over the weekend, Alyssa P. Hacker took a break from
studying to develop a 
piece of software that:
\begin{enumerate}
\item[1.] Scans the Internet for all IoT devices that are vulnerable to the
  Mirai botnet by attempting to log into each of the vulnerable devices; and
\item[2.] If the software succeeds in logging into the vulnerable device, it
  changes the password to the camera's hardware address (the hardware
  address is something that the owner of the camera could easily
  discover but, after the vulnerability is closed, would be relatively
  more difficult for remote attackers to discover).
\end{enumerate}

\prob{10} Explain why we need ethical reasoning guidelines such as those
from the Belmont Report, as opposed to simply encoding an ethical rule
or law that permits or prevents deploying this type of software.
\eprob

\sols{\textit{Answer:} Technology is evolving much more rapidly than rules can
be codified. For example, the Common Rule, governing Institutional Review
Boards at universities, dates to 1981. With ethical guidelines and principles,
we can continue to reason about ethics in the face of new and unforseen
technologies and advancements.}

\newpage
\prob{30} Alyssa is considering whether to deploy her code to patch the
Internet-wide vulnerability.  Reason about the ethics of deploying Alyssa's
patch in terms of each of the following:
\begin{enumerate}
\item Respect for persons
\item Beneficence
\item Justice
\end{enumerate}
\eprob

\sols{
\textit{Answer:}
	\begin{itemize}
		\itemsep=-1pt
		\item Respect for persons: Some discussion of informed consent would have been
		the most straightforward way to address this question. One way to ensure that
		humans agree to participate in an activity or experiment is to directly ask them.
		\item Beneficience: This factor involves a consideration of whether risks outweigh
		benefits. A discussion of some of the risks of the activity (e.g., inadvertently
		bricking the devices, causing reboots at inconvenient times) also could relate
		to respect for persons, but beneficence would need to explicitly consider whether
		these risks are outweighed by the benefits of securing the devices.
		\item Justice: This principle involves determining whether the parties that bear
		the risks associated with the activity are the same as those who reap the benefits.
		Determining the ``right'' answer here is not straightforward, since it requires
		reasoning about who might reap the benefits of the activity. On the one hand it
		might be others who might become victims of a DoS attack; on the other, the users
		themselves might also be beneficiaries of the improved network hygiene, as well
		(no rogue devices consuming bandwidth on their network, for example). The key
		point of the discussion here was to discuss equitable distribution of risks.
	\end{itemize}
}
