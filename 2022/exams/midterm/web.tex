\pagebreak
\section{Web Security}

\prob{4} 
Which of the following are true about cross-site scripting (XSS) attacks?
\allapply

\setcounter{partctr}{0}
\begin{list}{\bf\Alph{partctr}.}{\usecounter{partctr}}
\item An attacker can use an XSS attack to steal a victim's Web cookies
  for another Web site.
\item A user can defend against XSS attacks by preventing the browser
  from executing any scripts.
\item A Web site can defend against an XSS attack by ensuring that the
  input to a script does not itself contain any code.
\item An attacker could launch an XSS attack by posting a comment on a
  message board.
\item All of the above.
\end{list}
\eprob

\sols{
\begin{answer}
The answer is: (E).
\end{answer}
}

\prob{6} Suppose an attacker is running a Web server that runs a script
that takes user and password input from a form and retrieves the
information for that user, as follows:

\begin{verbatim}
set ok = execute( "SELECT * FROM Users
	   WHERE user=' "  &  form(“user”)  & " ' 
           AND   pwd=' " & form(“pwd”) & “ '” );
if not ok.EOF   
           login success  
else  fail;
\end{verbatim}

\setcounter{partctr}{0}
\begin{list}{\bf\Alph{partctr}.}{\usecounter{partctr}}
\item Write down input to the ``user'' part of the form that would
  cause the database to reveal every row of the table (i.e., information
  for all users).
\item Write down input to the ``user'' part of the form to cause the
  entire user table to be deleted.
\end{list}
\eprob
~\ansbelow



