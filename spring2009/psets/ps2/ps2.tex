\documentclass[11pt]{article}

\usepackage{epsf}
\usepackage{epsfig}
\usepackage{url}
\usepackage{color}
\usepackage{6829hw}
\input{macros.tex}

\begin{document}

\newcounter{listcount}
\newcounter{sublistcount}


\handout{PS2}{March 2, 2009}{Instructor: Prof. Nick Feamster}
{College of Computing, Georgia Tech}{Problem Set 2: Crypto and Sockets}

This problem set has three questions, each with several parts.  Answer
them as clearly and concisely as possible.  You may discuss ideas with
others in the class, but your solutions and presentation must be your
own.  Do not look at anyone else's solutions or copy them from
anywhere. (Please refer to the Georgia Tech honor code, posted on the
course Web site).

Turn in your solutions in on {\bf March 16, 2009} by 11:59 p.m. -0500,
to the grader.

Note: Please use either C or C++ for Q2 and Q3, and make sure that it compiles
using a recent version of gcc/g++. Apart from correctness of the result,
the running time of your programs in Q2 and Q3 (part 1) also affects
your grade. The grade for a program, 'g' will be the sum of three
components:
\begin{itemize}
\item  'i' - for your implementation
\item  'c' - functionality (i.e., correctness)
\item  't' - running time, a weighted score, with the submission with the
        best running time receiving the full 't' points.
\end{itemize}
\noindent
Please make sure your programs have a batch mode that lets us measure
the runtime. {\em Your code must support the invocations as specified on
  the class wiki.} \\ \url{
  http://davis.gtnoise.net/wiki/doku.php?id=cs6262:problem_set_2_updates}

\section{Warm Up}

Suppose that you have a hash function $H$, where the probability of
collision is $2^{-20}$.  You are designing a network protocol that takes
the hash of each packet header as inputs.  Your goal is to design a
protocol so that each packet produces a unique ``signature''; every
router along a path that hashes that packet should produce the same
signature.

\begin{enumerate}
\item Which fields in the packet header would you choose as input?
  Which ones might pose problems for achieving the above goals (think
  about packet headers that may be modified in-flight)?  Hint: You need
  to select headers that include both unique headers for each packet,
  and are invariant across paths.
\item Suppose that you have a flow that is $n$ packets.  Assuming a
  uniform hash function, what is the probability that two packets in
  that flow will produce the same output (i.e., what is the probability
  that a collision will occur within the first $n$ packets).  Express
  the collision probability as a function of $n$.
\item To reduce the probability of collision, you might consider
  producing the packet signature using $k$ independent uniform hash
  functions (as is done with a Bloom filter).  Given $k$ independent
  hash functions, what is the probability that two packets in an
  $n$-packet stream will produce the same signature?
\end{enumerate}

\section{Fast Exponentiation}

Your implementation for this problem should be in C (or C++).  

Implement a program that computes exponents efficiently, using the
mechanism described in the textbook. Your program should be able to deal
with exponents of up to 32 bits and modulo value of up to 32 bits. Make
use of the long long 64-bit arithmetic found on most modern
machines. Your program should take base, exponent and modulus as
arguments, such as

\begin{verbatim}
  exp 45678 123 6789
\end{verbatim}

Please print out the running time of your operations.    

\section{Ciphers}

Your implementation for this problem should be in C (or C++).

\begin{enumerate}
\item Implement one round of DES and one round of RC5 (for RC5, use a
  64-bit key and 64-bit blocks).  Turn in the source code for your
  implementation of each stream cipher, including compilation and
  running instructions.  

\item Use a profiler to examine the complexity of each of the operations
  in the stream cipher (e.g., substitution, swapping, etc.).  How much
  time is spent in each of the operations, for each of the stream
  ciphers?  For this part of the problem, please include a table that
  breaks down the operation of the ciphers, with a percentage of CPU
  time spent in each part of the process.
\item One of the major design motivations for DES, in particular, was
  that certain operations could be performed more efficiently in
  hardware than in software.  In the case of DES, briefly explain which
  operations could be made more efficient in hardware, and why they
  could be more efficient in hardware than in software (Hint: think
  about parallelization, for example.)
\end{enumerate}

\section{OpenSSL Programming}

In this problem, you will implement a public-key-based secure channel
using OpenSSL.  For extra credit, you can perform some extensions to the
sockets layer itself.

\begin{enumerate}
\item Download and install the openssl library.
\item Write a simple client and server that uses RSA public key
  encryption between the client and server.  
\item Modify your client and server to use an elliptic-curve based
  public-key algorithm (this should be a small modification).  Note that
  the size of the public keys are much smaller in this case.  What
  advantages can you think of for using smaller keys?
\item {\em Extra credit:} Recall that, in lecture, we discussed the
  benefits of self-certifying protocols over standard key distribution
  mechanisms.  Modify your client/server setup so that the client can
  communicate with the server using self-certifying techniques.

  {\em Hint:} You might do this in a number of ways.  You could modify
  the lookup service, hack the IP layer and use proxies, etc.  Be
  creative, and state your assumptions. 
\end{enumerate}

\newpage
\section*{Quiz Review Topics}

For your benefit, here's a rough list of topics that should help you in
studying for the quiz.  This is not necessarily a complete list, but it
should serve as a fairly good starting point for studying.

\begin{enumerate}
%
\item General Security Concepts
\begin{itemize}
\item definitions: confidentiality, authentication, authorization, etc.
\item threat taxonomies and attack models
\end{itemize}
%
\item Cryptography
\begin{itemize}
\item secret key ciphers: substitution, permutation, combinations
  thereof
\item historical secret key ciphers
\item security arguments for one-time pad
\item block and stream ciphers
\item public key algorithms: Diffie-Hellman, RSA
\item fast exponetiation techniques
\item operation of RSA public key algorithm
\item types of attacks: dictionary, chosen plaintext, known plaintext, etc.
\item attacks on RSA (those covered in lecture)
\end{itemize}
%
\item Key Management
\begin{itemize}
\item key distribution centers
\item public key infrastructures: design, drawbacks, etc.
\item Kerberos: roughly how it works
\end{itemize}
\item Authentication and Access Control
\begin{itemize}
\item Lattice models
\item access control: matrices, capabilities, ACLs
\item taint propagation
\item static vs. dynamic tainting
\end{itemize}
\item System Security 
\begin{itemize}
\item Why crypotsystems fail
\item Secure fault localization
\end{itemize}
\item Malware and Worms
\begin{itemize}
\item buffer overflow attacks: stack smashing, off-by-one, format string
\item SQL injection attacks
\item work propagation: models, scanning strategies, etc.
\end{itemize}

\end{enumerate}

\end{document}
