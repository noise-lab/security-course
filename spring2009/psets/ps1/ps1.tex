\documentclass[11pt]{article}

\usepackage{epsf}
\usepackage{epsfig}
\usepackage{url}
\usepackage{color}
\usepackage{6829hw}

\newcommand{\newc}{\newcommand}

\newc{\code}[1]{{\tt #1}}
\newc{\func}[1]{{\em #1\/}}

\newc{\be}{\begin{enumerate}}
\newc{\ee}{\end{enumerate}}

\newc{\bi}{\begin{itemize}}
\newc{\ei}{\end{itemize}}

\newc{\bd}{\begin{description}}
\newc{\ed}{\end{description}}

\newc{\ov}[1]{$\overline{#1}$}
\newc{\instr}{\tt}

\newc{\doublespace}{\renewcommand{\baselinestretch}{1.5}}

\newcommand{\figref}[1]{Figure~\ref{#1}}
\newcommand{\tref}[1]{Table~\ref{#1}}

% Captioned table
\newc{\tbl}[3]{
        \begin{table}[htb]
                \centering
                #1
                \caption{#3}
                \label{#2}
        \end{table}
}

% Input a table.
\newcommand{\dblfig}[3]{
        \begin{figure}[htb]
		\centering
                \input{#1}
                \caption{#3}
                \label{#2}
        \end{figure}
}

\newcommand{\ddblfig}[4]{
        \begin{figure}[htb]
		\hspace{-0.1in}
                \psfig{figure=#1,width=0.45\textwidth}
                \caption{#3}
                \label{#2}
        \end{figure}
}

% Figure with no caption
\newcommand{\nofig}[2]{
        \begin{figure}[htb]
                \centering
                \psfig{figure=#1}
                \label{#2}
        \end{figure}
}

% Whole page figure
\newcommand{\schfig}[3]{
        \begin{figure}[p]
                \centering
                \psfig{figure=#1,height=7in}
                \caption{#3}
                \label{#2}
        \end{figure}
}

% Small figure
\newcommand{\sfig}[3]{
        \begin{figure}[ltb]
                \centering
               \hspace*{\fill}\rule{\linewidth}{.5mm}\hspace*{\fill}\vspace{3mm}
                \psfig{figure=#1,width=0.4\textwidth}
                \caption{#3}
                \label{#2}
               \vspace{3mm}\hspace*{\fill}\rule{\linewidth}{.5mm}\hspace*{\fill}
        \end{figure}
}

% Medium figure
\newcommand{\mfig}[3]{
        \begin{figure}[ltb]
		\centering
               \hspace*{\fill}\rule{\linewidth}{.5mm}\hspace*{\fill}\vspace{1mm}
                \psfig{figure=#1,height=2.5in}
                \caption{#3}
                \label{#2}
               \vspace{0mm}\hspace*{\fill}\rule{\linewidth}{.5mm}\hspace*{\fill}
        \end{figure}
}

\newcommand{\widefig}[4]{
        \begin{figure*}[htb]
                \centering
               \hspace*{\fill}\rule{\linewidth}{.5mm}\hspace*{\fill}\vspace{5mm}
                \psfig{figure=#1,width=#3}
                \caption{#4}
                \label{#2}
               \vspace{5mm}\hspace*{\fill}\rule{\linewidth}{.5mm}\hspace*{\fill}
        \end{figure*}
}

\newcommand{\mcfig}[4]{
        \begin{figure}[htbp]
                \centering
               \hspace*{\fill}\rule{\linewidth}{.5mm}\hspace*{\fill}\vspace{5mm}
                \psfig{figure=#1,width=#3}
                \caption{#4}
                \label{#2}
               \vspace{5mm}\hspace*{\fill}\rule{\linewidth}{.5mm}\hspace*{\fill}
        \end{figure}
}

\newcommand{\docfig}[3]{
        \begin{figure}[htbp]
               \hspace*{\fill}\rule{\linewidth}{.5mm}\hspace*{\fill}\vspace{5mm}
                \centering
                \psfig{figure=#1,width=#3}
                \label{#2}
               \vspace{5mm}\hspace*{\fill}\rule{\linewidth}{.5mm}\hspace*{\fill}
        \end{figure}
}

% Medium-large figure
\newcommand{\mlfig}[3]{
        \begin{figure}[htb]
                \centering
               \hspace*{\fill}\rule{\linewidth}{.5mm}\hspace*{\fill}\vspace{5mm}
                \psfig{figure=#1,height=3.25in}
                \caption{#3}
                \label{#2}
               \vspace{5mm}\hspace*{\fill}\rule{\linewidth}{.5mm}\hspace*{\fill}
        \end{figure}
}

% Large figure
\newcommand{\lfig}[3]{
        \begin{figure}[p]
                \centering
               \hspace*{\fill}\rule{\linewidth}{.5mm}\hspace*{\fill}\vspace{5mm}
                \psfig{figure=#1,height=5in}
                \caption{#3}
                \label{#2}
               \vspace{5mm}\hspace*{\fill}\rule{\linewidth}{.5mm}\hspace*{\fill}
        \end{figure}
}

% 'gg' figures are the double column versions of the 'g' figures above.
\newcommand{\sfigg}[3]{
        \begin{figure*}[htb]
                \centering
               \hspace*{\fill}\rule{\linewidth}{.5mm}\hspace*{\fill}\vspace{5mm}
                \psfig{figure=#1,height=1.5in}
                \caption{#3}
                \label{#2}
               \vspace{5mm}\hspace*{\fill}\rule{\linewidth}{.5mm}\hspace*{\fill}
        \end{figure*}
}

% Medium figure
\newcommand{\mfigg}[3]{
        \begin{figure*}
                \centering
               \hspace*{\fill}\rule{\linewidth}{.5mm}\hspace*{\fill}\vspace{5mm}
                \psfig{figure=#1,width=\linewidth}
                \caption{#3}
                \label{#2}
               \vspace{0mm}\hspace*{\fill}\rule{\linewidth}{.5mm}\hspace*{\fill}
        \end{figure*}
}

% Medium-large figure
\newcommand{\mlfigg}[3]{
        \begin{figure*}[htb]
                \centering
               \hspace*{\fill}\rule{\linewidth}{.5mm}\hspace*{\fill}\vspace{5mm}
                \psfig{figure=#1,height=3.25in}
                \caption{#3}
                \label{#2}
               \vspace{5mm}\hspace*{\fill}\rule{\linewidth}{.5mm}\hspace*{\fill}
        \end{figure*}
}

% Large figure
\newcommand{\lfigg}[3]{
        \begin{figure*}[p]
                \centering
               \hspace*{\fill}\rule{\linewidth}{.5mm}\hspace*{\fill}\vspace{5mm}
                \psfig{figure=#1,height=5in}
                \caption{#3}
                \label{#2}
               \vspace{5mm}\hspace*{\fill}\rule{\linewidth}{.5mm}\hspace*{\fill}
        \end{figure*}
}

% Variable size figure
\newcommand{\vfigg}[4]{
        \begin{figure*}[htb]
                \centering
               \hspace*{\fill}\rule{\linewidth}{.5mm}\hspace*{\fill}\vspace{5mm}
                \psfig{figure=#1,#2}
                \caption{#4}
                \label{#3}
               \vspace{5mm}\hspace*{\fill}\rule{\linewidth}{.5mm}\hspace*{\fill}
        \end{figure*}
}

\newcommand{\vfig}[4]{
        \begin{figure}[ltb]
                \centering
               \hspace*{\fill}\rule{\linewidth}{.5mm}\hspace*{\fill}\vspace{1mm}
                \psfig{figure=#1,#2}
                \caption{#4}
                \label{#3}
               \vspace{1mm}\hspace*{\fill}\rule{\linewidth}{.5mm}\hspace*{\fill}
        \end{figure}
}

\newcommand{\vnlfig}[4]{
        \begin{figure}[htb]
                \centering
               \hspace*{\fill}\rule{\linewidth}{0mm}\hspace*{\fill}\vspace{5mm}
                \psfig{figure=#1,#2}
                \caption{#4}
                \label{#3}
               \vspace{0mm}\hspace*{\fill}\rule{\linewidth}{0mm}\hspace*{\fill}
        \end{figure}
}

\newcommand{\dblvfig}[6]{
        \begin{figure}[htb]
                \centering
                \hspace*{\fill}\rule{\linewidth}{0mm}\hspace*{\fill}\vspace{0.5mm}
                \psfig{figure=#1,#2}
	        \hspace{1in}
                \psfig{figure=#3,#4}
                \caption{#6}
                \label{#5}
               \vspace{2mm}\hspace*{\fill}\rule{\linewidth}{0mm}\hspace*{\fill}
        \end{figure}
}
\newc{\myspacing}{
        \let\oldtextheight=\textheight
        \let\oldtextwidth=\textwidth

        \let\oldtopmargin=\topmargin
        \let\oldheadheight=\headheight
        \let\oldfootheight=\footheight
        \let\oldheadsep=\headsep
        \let\oldoddsidemargin=\oddsidemargin


        \textheight 8.5in
        \textwidth 6in

        \topmargin 0in
        \headheight 0in
        \footheight 1.5in
        \headsep 0in
        \oddsidemargin 0in

}

\newc{\oldspacing}{
        \let\textheight=\oldtextheight 
        \let\textwidth=\oldtextwidth

        \let\topmargin=\oldtopmargin 
        \let\headheight=\oldheadheight 
        \let\footheight=\oldfootheight
        \let\headsep=\oldheadsep
        \let\oddsidemargin=\oldoddsidemargin
}
% Local Variables: 
% mode: latex
% TeX-master: t
% End: 


\begin{document}

\newcounter{listcount}
\newcounter{sublistcount}


\handout{PS1}{January 16, 2009}{Instructor: Prof. Nick Feamster}
{College of Computing, Georgia Tech}{Problem Set 1: Passwords and Packets}

This problem set has three questions, each with several parts.  Answer
them as clearly and concisely as possible.  You may discuss ideas with
others in the class, but your solutions and presentation must be your
own.  Do not look at anyone else's solutions or copy them from
anywhere. (Please refer to the Georgia Tech honor code, posted on the
course Web site).

Turn in your solutions in on {\bf January 31, 2009} by 11:59 p.m. -0500,
to the grader.

\section{Warm Up}

\begin{enumerate}
\item Examine the source code or man page for {\tt crypt}.  How does
  this program take a plaintext password and generate the ciphertext
  that we see in {\tt /etc/password}, or {\tt /etc/shadow}?  What cipher
  is used to generate the cipher from the plaintext?
%
\item Using {\tt crypt(3)} on a UNIX machine, generate the
  ciphertext for {\tt security} and {\tt netsecurity}.  What do you
  observe?  Why?
\item {\tt passwd} typically uses something called a ``salt'' to
  generate the password for each user.  Why?
\end{enumerate}

\section{Password Cracking}

For this problem, you will need to install John the Ripper:
\url{http://www.openwall.com/john/}. 

Consider the following password file generated with {\tt crypt(3)}:

\begin{verbatim}
root:IWpIzqD0jR1.c:100:100:Charlie Root:/home/root:/bin/sh
cs6262:UNzrFi5aYL9DU:101:101:CS6262:/home/cs6262:/bin/sh
mysql:WqCBVG36lcuAc:102:102:MySQL:/home/mysql:/bin/sh
guest:FTQinpjr.VRM.:103:103:Guest:/home/guest:/bin/sh
test:LF2c9qM5l6X7Q:104:104:Testing:/home/test:/bin/sh
\end{verbatim}

You can obtain this password file from \\
\url{http://www.gtnoise.net/classes/cs6262/spring_2009/psets/ps1/6262.pw}. 

\begin{enumerate}
\item Run the default mode of John; one of these passwords will be
  cracked.  Which one?  Why (which rule of John was applied)?

\item Now, try the ``wordlist'' mode of John.  Which password is cracked
  now?  Which rule of John was applied?  ({\em Hint:} John's default
  wordlist is quite small by default.  You may have to augment this
  wordlist with one of your own.

\item One of the users has a pw that is a rotated version of a
dictionary word. Modify John's rule list to incorporate this feature.
Which password does this now reveal?  Please include the source for your
modifications to the rule list.

%% For this, they'll have to read john's rule syntax. It's easy enough if
%% they look through the default config file. There's a rule in the
%% single-cracking mode (without wordlists) that tries precisely this,
%% but because the single mode does not use dictionary words, it cant do
%% the crack. All they need to do is to find the appropriate rule and
%% copy it to another section of the config file.

\item One user is predisposed to using leetspeak (4 for a/A, 1 for i/I
and l, 3 for e/E). His password is also a dictionary word. Modify
john.conf to incorporate this feature.  Which password is revealed?

%% Again, the rule exists in john.conf's single-crack section, and all
%% they need to do is copy it to the wordlist section. The key here is
%% that they can't blindly copy the existing leetspeak rules; the
%% existing rules don't have a substitution for '1' for 'i'. So they'll
%% have to change it a little bit.

\item One user likes to swap two adjacent characters of a dictionary
word. Can you modify john.conf to do this using existing syntax? If not,
how can you incorporate this feature?  What is the password that is
revealed?

%% So, from looking at the rule syntax, this is not possible straight off
%% the bat. There're a few ways to do this. One is to read in the
%% wordlist and generate all possible combinations from a given wordlist
%% (e.g., for 'toy', there'd now be 'toy', 'oty', and 'tyo'). But the
%% much easier alternative is to add a new rule to john by changing the
%% source. I just figured this out myself, and it took just 7 lines to
%% add the rule. Pretty easy, but gives people practice of reading others
%% code and finding where to add stuff. We can probably add a "Hint: you
%% can also edit the source code" if it seems too challenging.

%% This should take someone about 3-4 hours starting from scratch. The
%% key would, of course, be to read the john docs and such.
\end{enumerate}

\section{TCP Stream Reassembly}

\begin{enumerate}
\item Write a program to reassemble an HTTP session. Your program should
display for each HTTP session the request and response between the
client and the server.  It should also output the reassembled HTML
response as a file.  A test trace is available here: \\
\url{http://www.gtnoise.net/classes/cs6262/spring_2009/psets/ps1/6262.pcap.bz2}.  
{\em We will grade with a different trace.}

Your program should produce the reassembled HTML: the TCP payload data,
without the headers.  Your reassembled stream should display properly
when opened in a browser like Firefox.  Here are the steps we will use
for grading:

\begin{itemize}
\item We will run the code you turn in against our test trace.
\item We will attempt to open the reassembled stream in Firefox.
\end{itemize}

{\em Hint:} you can get the source code of the tcpdump and modify it to
record the payload of TCP packets of HTTP sessions. 

\item Suppose that a TCP stream reassembler has been installed as part
  of a firewall, $n$ hops upstream from a server that it is trying to
  pretect.  Suppose also that your TCP stream reassembler can reassemble
  a telnet stream and is looking for the password ``{\tt password}'' in
  an incoming TCP stream.  

\setcounter{listcount}{0}
\begin{list}{(\alph{listcount})}{\usecounter{listcount}}
\item Explain how you might construct a TCP stream that transmits the
  entire stream {\tt password} to the server but prevents the
  reassembler from raising an alarm.  Note that an improved TCP stream
  reassembler would be able to reassemble strings across packets.  ({\em
  Hint:} Think about how you might get a packet to reach the reassembler
  but not the server.)
\item {\bf Extra credit. } Write a program that implements your
  exploit.  
\end{list}

\end{enumerate}

Please turn in your code to this assignment to the grader, Priyank Raj,
by email.  Your code should include a README describing everything we
need to do to compile and run your code.


\end{document}