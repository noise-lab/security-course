%Checkpoint 2
%points
\newcounter{pts-cp2}
\newcounter{pts-cp2.1}
\newcounter{pts-wep}  \setcounter{pts-wep}{15}
\newcounter{pts-clientip} \setcounter{pts-clientip}{5}
\newcounter{pts-serverip} \setcounter{pts-serverip}{5}
\newcounter{pts-protocol} \setcounter{pts-protocol}{12}
\newcounter{pts-priv} \setcounter{pts-priv}{5}
\newcounter{pts-cred} \setcounter{pts-cred}{20}
\newcounter{pts-secret} \setcounter{pts-secret}{6}
\newcounter{pts-jailtime} \setcounter{pts-jailtime}{2}
\setcounter{pts-cp2.1}{
    \value{pts-wep}+\value{pts-clientip}+\value{pts-serverip}+\value{pts-protocol}+\value{pts-priv}+\value{pts-cred}+\value{pts-secret}+\value{pts-jailtime}
}
\newcounter{pts-cp2.2} \setcounter{pts-cp2.2}{30}
\setcounter{pts-cp2}{\value{pts-cp2.1}+\value{pts-cp2.2}}

\section{Checkpoint 2 (\arabic{pts-cp2} points)}
\label{sec:cp2}

\subsection{Network Attacks (\arabic{pts-cp2.1} points)}
\label{sec:cp2.1}

In this part of the project, you will experiment with network attacks by:

\begin{itemize}[nosep]
  \item cracking password for a WiFi network protected with WEP (Wired Equivalent Privacy),\\a security protocol designed for WiFi networks
  \item analyzing network traffic
  \item obtaining a victim's credentials
\end{itemize}

\medskip

At Siebel Center Room 1129 (CS460/ECE419 Lab), you will find a WEP-encrypted WiFi network named \texttt{cs461fa16}.
We've created this network specifically for you to attack, and you have permission to do so.
On the network, you will find a server and client(s), all wirelessly connected.

\medskip

Your goal is to obtain a client's secret retreived from the server.

\medskip

You SHOULD use suggested tools to complete the tasks.
If you need to use anything else, you MUST ask a TA for permission.

\medskip

If you run into issues due to having incompatible wireless adapter, you MAY rent a USB wireless adapter during TA's office hours.
Rent period is 24 hours.
If returning early or during weekends, please arrange a meeting with a TA by sending an email to \staffemail.

\hypertarget{cp2setup}{\subsubsection*{Setup}}
\begin{itemize}
  \item \textbf{For Mac (OS X) Users}\\
  We \textbf{STRONGLY recommend you to find a partner who has a non-Apple laptop}.
%  If you choose to use OS X to do this assignment, you have following two options, which may or may not work:
%  \begin{itemize}[nosep]
%    \item Use the built-in AirPort Utility on OS X
%      \begin{itemize}[nosep]
%        \item You can use AirPort Utility, in place of Airmon-ng and Airodump-ng (tools included in the Aircrack-ng Suite), to capture packets in monitor mode.
%        \item Limitation: AirPort Utility does not let you specify the wireless network you are interested in capturing. You may have to wait for hours, or even days depending on the volume of wireless traffic in the area, to capture enough packets to successfully crack the network key.
%      \end{itemize}
%    \item Use Kali Linux Live and USB wireless adapter
%      \begin{itemize}[nosep]
%        \item As it is difficult to find stable wireless card drivers for Linux running on Apple devices, you will have to use a USB wireless adapter that is compatible with the Aircrack-ng Suite.
%        \item Limitation: We have a very limited number of USB wireless adapters available for rent.
%      \end{itemize}
%  \end{itemize}

  \item \textbf{For Others}\\
  We \textbf{STRONGLY recommend using Kali Linux Live booted from a USB drive}.
  \begin{enumerate}[nosep]
    \item Download Kali Linux Light 32 bit:\\
    \url{http://cdimage.kali.org/kali-2016.2/kali-linux-light-2016.2-i386.iso}
    \item Create a Kali Live Bootable USB drive:\\
    \url{http://docs.kali.org/downloading/kali-linux-live-usb-install}
    \item Add persistence to your drive to allow storing your data and configuration:\\
    \url{http://docs.kali.org/downloading/kali-linux-live-usb-persistence}
    \item Boot from your USB drive, then choose "Live USB Persistence" from the menu.
    \item Install Wireshark:
    \vspace{-4pt}
    \begin{mdframed}
    \begin{Verbatim}
apt-get update
apt-get install wireshark
    \end{Verbatim}
    \end{mdframed}
    \end{enumerate}
  \textbf{Warning}: Use the Kali USB drive on your own computers only. DO NOT use the drive on school computers, such as EWS or any other lab machines.
%  \item \textbf{Do Not Use VMs}\\
%  We \textbf{DO NOT recommend using virtual machines}.
%  Your built-in wireless card may not properly work with Aircrack-ng in virtual machines.
%  If you choose this option, we cannot help you with troubleshooting.

\end{itemize}

\pagebreak

\subsubsection{WEP cracking (\arabic{pts-wep} points)\hfill\rm\normalsize (\emph{Difficulty: Medium})}
\label{ssec:wep}
\newcommand{\filewep}{\ref*{ssec:wep}.txt}

First, you will need to crack the WEP encryption key for the network using Aircrack-ng Suite. Aircrack-ng website provides many helpful tutorials (e.g. \url{http://www.aircrack-ng.org/doku.php?id=simple\_wep\_crack}). You SHOULD go through the tutorials to understand the purpose of each tool and learn how to use them.

\smallskip

As you will learn from the tutorial, WEP cracking process usually involves performing certain attacks to generate traffic so that data can be collected faster.
For this project, however, you MUST NOT use those attacks as they severly disrupt the network.

\smallskip

We've made sure that there's sufficient traffic generated on the network to allow successful WEP cracking within a reasonable amount of time (< 1 hour).
If the process is taking too long, please notify a TA.

\smallskip

\hypertarget{cp2wep}{\textbf{What to submit}\hspace{10pt}}
Submit \texttt{\hyperlink{wepformat}{\filewep}} that contains the WEP key in \textit{ASCII characters}.

\subsubsection{Client IP address (\arabic{pts-clientip} points)\hfill\rm\normalsize (\emph{Difficulty: Medium})}
\label{ssec:clientip}
\newcommand{\fileclientip}{\ref*{ssec:clientip}.txt}

Examine the network traffic with Wireshark and identify the IP address(es) of the client(s).

\textbf{Note:} Other students (and yourself) on the network are not considered as clients.

\smallskip

\hypertarget{cp2clientip}{\textbf{What to submit}\hspace{10pt}}
Submit \texttt{\hyperlink{clientipformat}{\fileclientip}} that contains the IP address(es).
Write one address per line.

\subsubsection{Server IP address (\arabic{pts-serverip} points)\hfill\rm\normalsize (\emph{Difficulty: Medium})}
\label{ssec:serverip}
\newcommand{\fileserverip}{\ref*{ssec:serverip}.txt}

Examine the network traffic with Wireshark and identify the IP address of the server.

\textbf{Note:} We consider a host as a server only if it provides services to other hosts. Other students (and yourself) on the network are not considered as servers.

\smallskip

\hypertarget{cp2serverip}{\textbf{What to submit}\hspace{10pt}}
Submit \texttt{\hyperlink{serveripformat}{\fileserverip}} that contains one IP address.

\subsubsection{Services (\arabic{pts-protocol} points)\hfill\rm\normalsize (\emph{Difficulty: Medium})}
\label{ssec:protocol}
\newcommand{\fileprotocol}{\ref*{ssec:protocol}.txt}

Find the services (below port 4096) provided by the server you identified in \ref{ssec:serverip}.\\
In addition to examining the network traffic with Wireshark, consider using nmap, a powerful tool for probing remote hosts, as well as interacting with the server yourself.

\smallskip

\hypertarget{cp2protocol}{\textbf{What to submit}\hspace{10pt}}
Submit \texttt{\hyperlink{protocolformat}{\fileprotocol}} that contains names of application layer protocols used by the services (in standard acronyms).
Write one protocol per line.

\subsubsection{Server's secret (\arabic{pts-priv} points)\hfill\rm\normalsize (\emph{Difficulty: Easy})}
\label{ssec:priv}
\newcommand{\filepriv}{\ref*{ssec:priv}.txt}

Find the \textit{server}'s secret file using the services you found in \ref{ssec:protocol}.

\smallskip

\hypertarget{cp2priv}{\textbf{What to submit}\hspace{10pt}}
Submit the secret file renamed as \texttt{\hyperlink{privformat}{\filepriv}}.

\pagebreak

\subsubsection{Client's credentials (\arabic{pts-cred} points)\hfill\rm\normalsize (\emph{Difficulty: Hard})}
\label{ssec:cred}
\newcommand{\filecred}{\ref*{ssec:cred}.txt}

The client(s) retrieve secret messages from the server using credentials that expire in an hour.
Obtain the credentials.
You may find multiple pairs; pick any one of them.

\textbf{Hint:} You may want to read up on the HTTP Basic Authentication method, which is specified in RFC~2617.

\textbf{Note:} We strongly recommend using 32 bit version of Wireshark for this part. Bugs within 64 bit version may prevent you from completing the task.

\smallskip

\hypertarget{cp2cred}{\textbf{What to submit}\hspace{10pt}}
Submit \texttt{\hyperlink{credformat}{\filecred}} that contains the username in first line and password in second.
You MUST submit only one pair.

\subsubsection{Client's secret message (\arabic{pts-secret} points)\hfill\rm\normalsize (\emph{Difficulty: Easy})}
\label{ssec:secret}
\newcommand{\filesecret}{\ref*{ssec:secret}.txt}

Obtain the client's secret message using the credentials you found in \ref{ssec:cred}.
Each retrieval attempt generates a different secret message; pick any one of them.

\smallskip

\hypertarget{cp2secret}{\textbf{What to submit}\hspace{10pt}}
Submit \texttt{\hyperlink{secretformat}{\filesecret}} that contains the secret message.
When obtaining the secret message, you MUST use the same credentials you submit for \ref{ssec:cred}.
You MUST submit only one.

\subsubsection{Jail time (\arabic{pts-jailtime} points)\hfill\rm\normalsize (\emph{Difficulty: Easy})}
\label{ssec:jailtime}
\newcommand{\filejailtime}{\ref*{ssec:jailtime}.txt}

What is the maximum number of years in jail that you could face under 18~USC~\S~2511 for intercepting traffic on an encrypted WiFi network without permission?

\smallskip

\hypertarget{cp2jailtime}{\textbf{What to submit}\hspace{10pt}}
Submit \texttt{\hyperlink{jailtimeformat}{\filejailtime}} that contains the maximum number of years in jail.
Write the number only.

\newpage

\subsection{Anomaly Detection (\arabic{pts-cp2.2} points)\hfill\rm\normalsize (\emph{Difficulty: Medium})}
\label{sec:cp2.2}
\newcommand{\fileanomaly}{\ref*{sec:cp2.2}.py}

In \ref{sec:cp1.1}, you manually explored a network trace and detected a port scanning activity.
Now, you will programmatically analyze trace data to detect such activity.

\medskip

Port scanning can be used offensively to locate vulnerable systems in preparation for an attack, or defensively for research or network administration.
In one port scan technique, known as a \texttt{SYN} scan, the scanner sends TCP \texttt{SYN} packets (the first step in the TCP handshake) and watches for hosts that respond with \texttt{SYN+ACK} packets (the second step).

\medskip

Since most hosts are not prepared to receive connections on any given port, typically, during a port scan, a much smaller number of hosts will respond with \texttt{SYN+ACK} packets than originally received \texttt{SYN} packets.
By observing this effect in a network trace, you can identify source addresses that may be attempting a port scan.

\medskip

Your task is to develop a Python program that analyzes a PCAP file in order to detect possible \texttt{SYN} scans.
You MUST use \texttt{dpkt} Python library for packet manipulation and dissection.\\
For more information about \texttt{dpkt}:
\begin{itemize}[nosep]
  \item PyDoc documentation - \texttt{pydoc dpkt}
  \item official website - \url{https://github.com/kbandla/dpkt}
\end{itemize}
You may also find this tutorial helpful:\\
\url{https://jon.oberheide.org/blog/2008/10/15/dpkt-tutorial-2-parsing-a-pcap-file}

\medskip

\textbf{Specifications}
\begin{itemize}
  \item Your program MUST take one argument, the name of the PCAP file to be analyzed:\\
    ex) \texttt{python2.7 \ref{sec:cp2.2}.py capture.pcap}
  \item Your program MUST print out the set of IP addresses (one per line) that sent more than 3 times as many \texttt{SYN} packets as the number of \texttt{SYN+ACK} packets they received.
  For the purpose of this assignment, this rule applies even if the number of packets is very small.
  For example, following cases are all considered as attacks:
\begin{mdframed}
\begin{verbatim}
SYN=4 ACK+SYN=1
SYN=4 ACK+SYN=0
SYN=1 ACK+SYN=0
\end{verbatim}
\end{mdframed}
  \item Each IP address MUST be printed only once.
  \item Your program MUST silently ignore packets that are malformed or that are not using Ethernet, IP, and TCP.
\end{itemize}

\pagebreak

\textbf{Example output:}
A sample PCAP file captured from a real network can be downloaded at:\\
\url{https://subversion.ews.illinois.edu/svn/fa16-ece422/\_shared/mp4/lbl-internal.20041004-1305.port002.dump.anon.pcap}

\begingroup\raggedleft
(Source: \url{ftp://ftp.bro-ids.org/enterprise-traces/hdr-traces05})\\
\endgroup

For this input, your program's output MUST be these lines, in any order:
\vspace{-6pt}
\begin{mdframed}
\begin{verbatim}
128.3.23.2
128.3.23.5
128.3.23.117
128.3.23.158
128.3.164.248
128.3.164.249
\end{verbatim}
\end{mdframed}

\smallskip

\hypertarget{cp2anomaly}{\textbf{What to submit}\hspace{10pt}}
Submit \texttt{\fileanomaly}, a Python program that accomplishes the task specified above.
You SHOULD assume that the grader uses \texttt{dpkt} 1.8.
You MAY use standard Python system libraries, but your program SHOULD otherwise be self-contained.
We will grade your detector using a variety of different PCAP files.

\newpage

\section*{Checkpoint 2: Submission Checklist}
\label{sec:cp2checklist}

Inside your mp4 SVN repository, you will have auto-generated files named as below.\\
Make sure that your answers for all tasks up to this point are submitted in the following files by \textbf{\duedate} at \textbf{\duetime}.

\subsection*{SVN Repository}
\nolinkurl{\svnrepo}

\subsection*{Team Members}
\texttt{partners.txt} : a text file containing NETIDs of both members, one NETID per line.\\
Place the NETID of the student making the submission in the first line.
\vspace{-12pt}
\hypertarget{cp2partners}{}
\subsubsection*{example content of \texttt{partners.txt}}
\begin{mdframed}
\begin{Verbatim}
your_netid
partner_netid
\end{Verbatim}
\end{mdframed}

\subsection*{Solution Format}

\hypertarget{wepformat}{}
\subsubsection*{example content of \texttt{\hyperlink{cp2wep}{\filewep}}}
\begin{mdframed}
\begin{Verbatim}
cs46!
\end{Verbatim}
\end{mdframed}

\hypertarget{clientipformat}{}
\subsubsection*{example content of \texttt{\hyperlink{cp2clientip}{\fileclientip}}}
\begin{mdframed}
\begin{Verbatim}
1.2.3.4
127.0.0.1
8.8.8.8
\end{Verbatim}
\end{mdframed}

\hypertarget{serveripformat}{}
\subsubsection*{example content of \texttt{\hyperlink{cp2serverip}{\fileserverip}}}
\begin{mdframed}
\begin{Verbatim}
1.2.3.4
\end{Verbatim}
\end{mdframed}

\hypertarget{protocolformat}{}
\subsubsection*{example content of \texttt{\hyperlink{cp2protocol}{\fileprotocol}}}
\begin{mdframed}
\begin{Verbatim}
ftp
telnet
smtp
pop3
\end{Verbatim}
\end{mdframed}

\pagebreak

\hypertarget{credformat}{}
\subsubsection*{example content of \texttt{\hyperlink{cp2cred}{\filecred}}}
\begin{mdframed}
\begin{Verbatim}
username1
p@ssw0rd123
\end{Verbatim}
\end{mdframed}

\hypertarget{secretformat}{}
\subsubsection*{example content of \texttt{\hyperlink{cp2secret}{\filesecret}}}
\begin{mdframed}
\begin{Verbatim}
1suPerSECretMesSageNKJn23kjsdjnK+Lskvnlksdf10dm23/sdfinvkwer2ThisShouldBe
InASingleLine23+4/
\end{Verbatim}
\end{mdframed}

\hypertarget{jailtimeformat}{}
\subsubsection*{example content of \texttt{\hyperlink{cp2jailtime}{\filejailtime}}}
\begin{mdframed}
\begin{Verbatim}
10
\end{Verbatim}
\end{mdframed}

\subsection*{List of files that must be submitted for Checkpoint 2}

\begin{itemize}
\item \texttt{\hyperlink{cp2partners}{partners.txt}}
\item \texttt{\hyperlink{cp2wep}{\filewep}}
\item \texttt{\hyperlink{cp2clientip}{\fileclientip}}
\item \texttt{\hyperlink{cp2serverip}{\fileserverip}}
\item \texttt{\hyperlink{cp2protocol}{\fileprotocol}}
\item \texttt{\hyperlink{cp2priv}{\filepriv}}
\item \texttt{\hyperlink{cp2cred}{\filecred}}
\item \texttt{\hyperlink{cp2secret}{\filesecret}}
\item \texttt{\hyperlink{cp2jailtime}{\filejailtime}}
\item \texttt{\hyperlink{cp2anomaly}{\fileanomaly}}
\end{itemize}
