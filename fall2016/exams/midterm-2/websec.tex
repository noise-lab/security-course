\newpage
\section{Web Security}

Much of this problem should be familiar from Assignments 2 and 3. The
code snippets should also be familiar, but be careful, as there are a few small
changes.

\prob{10}
What is a CSRF attack? Explain the purpose of token validation that you learned
from the assignment.

\sols{
    These are the easiest questions. I think the homework for this part is either easy or related to XSS attack. So I just come up a question about definitions. 
}
\eprob
\vspace*{0.5in}

\prob{10} Consider a web server that is running the following PHP code to authenticate
users:
{\footnotesize
\begin{verbatim}
if (isset($_POST[’username’]) and isset($_POST[’password’])) {
    $username = $_POST[’username’];
    $password = $_POST[’password’];
    $sql_s = "SELECT * FROM users WHERE password='$password' and username='$username'";
    $rs = mysql_query($sql_s);
    if (mysql_num_rows($rs) > 0) {
        echo "Login successful!";
    } else {
        echo "Incorrect username or password";
    }
}
\end{verbatim}
}
Write an input pair (username, password) that bypasses authentication procedures
and logs in as a user named `admin'. Briefly explain your answer. (Hint: `and' has
a higher precedence than `or')
\eprob

\sols{
	username = admin; password = ' or ''='; It is almost the same as the homework but I simply change the order of username and password. 
}

\if 0
\prob{10} Consider a web server that uses the following PHP code to authenticate 
users:
{\footnotesize
\begin{verbatim}
if (isset($_POST[’username’]) and isset($_POST[’password’])) {
    $username = $_POST[’username’];
    $password = md5($_POST[’password’], false);
    $sql_s = "SELECT * FROM users WHERE username=’$username’ and pw=’$password’";
    $rs = mysql_query($sql_s);
    if (mysql_num_rows($rs) > 0) {
        echo "Login successful!";
    } else {
        echo "Incorrect username or password";
    }
}
\end{verbatim}
}

Note that this code uses {\tt false} as the second parameter of the {\tt
md5()} function, which cases the function to return the hash as a $32$-character
hexadecimal number in string. {\bf Is this code secure against any SQL
injection?} (Yes or no.) If so, explain why. If not, provide an input pair
(username, password) that bypasses authentication procedures by logging in as
a user named `admin'. In addition, suggest a solution to fix this code and
protect against any SQL injection if you think this code is vulnerable. (Hint:
\#, -- are two possible comment characters in MySQL)

\sols{
	The answer is NO. Let username = admin';\#  and password be an empty string. Prepared statement or escaping username (and password). 
}
\eprob

\prob{10}  Recall that for the XSS attack assignment, you were asked to
demonstrate XSS attacks against Bungle’s search box, which does not properly
filter search terms before echoing them to the results page. Your goal was to
construct a URL that, if loaded in the victim’s browser, correctly executes
the payload which required persistence as one of the requirements.

Provide a step-by-step explanation of how how you made your payload persist
when a user logged into Bungle with his account. (You are not required to
explain how to deal with the backward/forward buttons.) {\em Your answer should
be no longer than about five steps; one short phrase for each step is sufficient.}

\sols{
	\begin{itemize}
	\item  remove original functionality of login button
	\item  bind login button with a new onclick event handler
	\item  read user input using jQuery/javascript function from username and userpass fields\\
	\item  using \$.ajax or other equivalent function to submit a POST request\\
	\item  reload the page by calling function equivalent to proxy() from framework code
\end{itemize}
}

\eprob
\fi
