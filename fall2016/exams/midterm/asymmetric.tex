\newpage
\section{Asymmetric Cryptography}

\prob{10} {What is one disadvantage of asymmetric compared to symmetric cryptography?}\eprob
\vspace*{0.5in}


\prob{30}
Alice wants to send a message \textit{M} to Bob. Assume Alice and Bob
have securely distributed their public keys $P_A$ and $P_B$ to
each other. Private keys of Alice and Bob are $S_A$ and $S_B$
respectively. Design messages that Alice must send to meet the security
requirement below.

Notation: 
\begin{itemize}
\item $\{x\}_y$ (x is encrypted using key y) 
\item $A \xrightarrow{x} B$ (A sending x to B)
\end{itemize}

Example:
\begin{itemize}
\item $A \xrightarrow{\{M\}{S_A}} B$ The message
  \textit{M} is encrypted with Alice's private
  key $S_A$. The encrypted message is sent to Bob.
\end{itemize}

\begin{enumerate}

\item Using public key cryptography, design a message that enables Bob to
verify the message source, Alice, and preserves only integrity.


\item Using public key cryptography, design a message that protects only the
confidentiality of the message sent from Alice to Bob.

\item Using public key cryptography, design a message that enables Bob to
verify the message source, Alice, and when both integrity and
confidentiality are protected.

\end{enumerate}
\eprob




%\item {Client Alice, now, wants to send to Bob multiple messages. Assume that
%Alice has the public key of Bob \textit{P\_B}. The private key of Bob is
%\textit{S\_B}. Describe how the two parties can establish a secure connection.
%Does Diffie-Hellman key exchange assures a secure connection between two
%parties?}

