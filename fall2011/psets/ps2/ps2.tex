\documentclass[11pt]{article}

\usepackage{epsf}
\usepackage{epsfig}
\usepackage{url}
\usepackage{6829hw}
\input{macros.tex}

\begin{document}

\newcounter{listcount}
\newcounter{sublistcount}

\handout{H2}{September 1, 2011}{Instructor: Prof. Nick Feamster}
{College of Computing, Georgia Tech}{Problem Set 2: Software Vulnerabilities}

This problem set has three questions, each with several parts.  Answer
them as clearly and concisely as possible.  You may discuss ideas with
others in the class, but your solutions and presentation must be your
own.  Do not look at anyone else's solutions or copy them from
anywhere. (Please refer to the Georgia Tech honor code, posted on the
course Web site).

Turn in your writeup and talk on {\bf September 15, 2011} by 11:59pm.
{\em Please upload your solutions to T-Square.  Other forms of
  submission will not be accepted!}  We will be providing more
information about how to turn in your assignment as the due date
approaches.


\begin{enumerate}
\item {\bf 15 points} Pfleger and Pfleeger, Section 1.11, Exercise 14.
\item {\bf 15 points} Pfleeger and Pfleeger, Section 1.11, Exercise 20.
\item {\bf 20 points} The Ware report from 1970 emphasizes defenses for
  physical attacks and leakage points. In many of today’s computing
  systems, however, physical attacks are relatively uncommon compared
  with other kinds of attacks.
  \begin{itemize}
    \item Give three distinct reasons, with appropriate justifications,
      why this is so. 
    \item In the Checkoway {\em et al.} paper on automotive security,
      the authors point out that it is possible to compromise the
      automotive system even without physical access to the car itself.
      From the paper identify at least one flaw of each type that
      creates a vulnerability: (1)~design; (2)~implementation; (3)~process
  \end{itemize}
\item {\bf 20 points} Discuss Ken Thompson's {\em Reflections on
  Trusting Trust} article in the context of Java applets.  What, if
  anything, would need to be changed?  What about byte-code verifiers,
  interpreted languages, disassemblers, etc. might you have to consider? 
\item {\bf 30 points} 

print\_display.c:

\begin{verbatim}
#include <stdlib.h>
#include <stdio.h>

int main(){
  char display[512];

  strcpy(display, getenv("DISPLAY"));
  printf("Your display environment variable is %s\n", display);

  return(0);

}
\end{verbatim}

The system administrator accidentally set the suid (set user id) bit on
this program, which is owned by root and lives in /usr/bin.

\begin{itemize}
\item Describe the vulnerability present in this program.
\item Write a program to exploit this vulnerability and obtain the root
  shell.  Be clear and concise.  If you use a language other than C or
  Perl, please be extra clear.
\item Argue/demonstrate that your exploit works.
\item Rewrite print\_display.c to be safe.
\end{itemize}

\end{enumerate}


\end{document}
