\section*{Privacy Laws}

\prob{4} Under the California Privacy Rights Act (CPRA), which of the
following is required for compliance with opt-out regulations? (Select all
that apply.) \\
\correctanswercircle{A conspicuous "Do Not Sell or Share My Personal
Information" link on the homepage.} \\
\answercircle{A privacy notice in a physical store.} \\
\correctanswercircle{Compliance with Global Privacy Control (GPC) signals.} \\
\answercircle{Providing opt-out rights only to California residents.} \\
\eprob

\prob{4} Does the CPRA result in spillover effects for businesses operating
outside of California?  \\
\yesnoyes
\eprob

\prob{4} Which of the following would be considered a spillover effect of
privacy laws like the CPRA? (Select all that apply.) \\
\correctanswercircle{A non-California website implementing a "Do Not Sell or
Share My Personal Information" link.} \\
\answercircle{A physical store displaying privacy notices in compliance with
the CPRA.} \\
\correctanswercircle{A company applying CPRA opt-out rights to users in other
states.} \\
\answercircle{A website refusing all opt-out requests from non-Californians.}
\\
\eprob

\prob{4}
Explain how the concept of "spillover effects" in privacy laws like the CPRA
may influence businesses that operate outside of California and consumers
nationwide. Provide
examples to illustrate your response.
\answerbox{3}{Spillover effects occur when companies implement privacy
policies that extend beyond the requirements of the CPRA to simplify
compliance, benefiting consumers outside California. For example, a business
might allow all U.S. customers to opt out of data sharing to streamline
operations. This increases privacy protections for users nationwide.}
\eprob

\section*{Software Copyright}

\prob{4} In determining fair use for copyrighted software, which factor is
typically the least significant in court rulings? \\
\answercircle{The purpose and character of the use.} \\
\correctanswercircle{Whether the software is freely available online.} \\
\answercircle{The effect of the use on the market for the original.} \\
\answercircle{The amount of the work used.} \\
\eprob

\prob{4} Is reverse engineering for the purpose of software interoperability
generally protected under U.S. copyright law? \\
\yesnoyes
\eprob

\prob{4}
Provide a case example (e.g., Sega v. Accolade or Google v. Oracle) to support
your answer. \\
\answerbox{2.5}{In Sega v. Accolade, the court ruled that reverse engineering
to achieve interoperability was protected under fair use. This precedent
supports using copyrighted software to develop compatible products, provided
the use does not harm the market for the original software.}
\eprob

\prob{4} Which of the following is an example of copyright laws being used to
censor content? (Select one.)
\answercircle{A government blocking a website hosting politically sensitive
information.} \\
\correctanswercircle{A DMCA takedown request targeting a video critical of a
company.} \\
\answercircle{A social media platform removing posts violating community
guidelines.} \\
\answercircle{A country requiring platforms to store data locally to restrict
access.} \\
\eprob


\section*{Content Moderation}

\prob{4} Which of the following challenges is most commonly associated with
automated content moderation systems? \\
\correctanswercircle{Difficulty in detecting nuanced or contextual speech.} \\
\answercircle{Cost of implementation.} \\
\answercircle{Lack of integration with user interfaces.} \\
\answercircle{Difficulty in scaling to large platforms.} \\
\eprob

\prob{4} Under Section 230 of the Communications Decency Act, are platforms
generally legally responsible for content posted by their users under federal
criminal law?  \\
\yesnono
\eprob

\prob{4}
Discuss the scope of Section 230 of the Communications Decency Act, including
its protections and exceptions, and explain how it applies to federal criminal
law. \\
\answerbox{2.5}{In Sega v. Accolade, the court ruled that reverse engineering
to achieve interoperability was protected under fair use. This precedent
supports using copyrighted software to develop compatible products, provided
the use does not harm the market for the original software.}
\eprob

\section*{Censorship}

\prob{4} Which of the following are examples of friction-based censorship
methods? (Select all that apply.) \\
\correctanswercircle{Requiring users to log in with verified accounts to
access certain websites.} \\
\correctanswercircle{Imposing slow loading times for certain politically
sensitive content.} \\
\answercircle{Flooding social media with state-sponsored propaganda.} \\
\answercircle{Blocking all access to international news websites.} \\
\eprob

\prob{4} Which of the following are examples of flooding as a censorship
strategy? (Select all that apply.) \\
\correctanswercircle{Publishing large amounts of irrelevant content to drown
out dissenting opinions.} \\
\answercircle{Requiring VPNs to access social media platforms.} \\
\correctanswercircle{Creating numerous fake social media accounts to amplify
state-sponsored narratives.} \\
\answercircle{Filtering search engine results to hide critical information.}
\\
\eprob

\prob{4} When using a VPN, which parties can potentially see your original
(and possibly unencrypted) traffic and your identity? (Select all that apply.)
\\
\correctanswercircle{The VPN provider.} \\
\correctanswercircle{Your device's local network administrator.} \\
\answercircle{The websites you visit after connecting to the VPN.} \\
\correctanswercircle{Government agencies monitoring the VPN provider.} \\
\eprob

\prob{4}
Explain how government censorship via filtering can inadvertently result in
citizens gaining more access to information. Reference specific mechanisms
discussed in class. \\
\answerbox{3}{Spillover effects occur when companies implement privacy
policies that extend beyond the requirements of the CPRA to simplify
compliance, benefiting consumers outside California. For example, a business
might allow all U.S. customers to opt out of data sharing to streamline
operations. This increases privacy protections for users nationwide.}
\eprob

\section*{Course Feedback}
\vspace*{-0.1in}
\prob{1}
Interest (1=Boring!; 10=Amazing!):
\shortanswerbox{0.5}{5}
Difficulty (1=Too easy; 10=Too hard):
\shortanswerbox{0.5}{5}
\eprob
\prob{3}
1. One thing you like. 2. One suggestion for improvement. 3. One thing you'd
like to see covered that wasn't covered:

\answerbox{3}{More free food.}
\eprob

