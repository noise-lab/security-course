\section*{Use Cases}

\prob{3} You come upon a network traffic trace that contains Internet
performance measurements (``speed tests'') from a large collection of users'
home routers, which were taken without their permission. The dataset contains
valuable information about Internet speeds around the world, but it also
contains personally identifiable information (PII) about the users. You are
confronted with the choice of whether to use this dataset to write a white
paper about Internet speeds in the United States. Use the three principles of
the Belmont Report to discuss the tradeoffs of using (or ignoring) this dataset. 


\answerbox{4}{}
\eprob

\prob{4} Describe at least one advantage of using an OAuth 2.0 authentication
infrastructure, versus using an API key

\answerbox{4}{}
\eprob

\section*{Data Acquisition}

\prob{2} Suppose you want to extract all Netflix traffic from a traffic
capture.  Capturing all traffic to and from the IP address for {\tt
netflix.com}  will yield all Netflix traffic streams.
\framebox{
\yesnono
}
\eprob

\prob{2}
In class, we used the domain name system (DNS) lookup traffic to identify Netflix
traffic. This approach can work in practice but is imperfect. List one reason
why DNS names may not always be practical for identifying traffic for services
like Netflix. {\bf (Answer box on next page!)}

\answerbox{1}{Domain names can change over time.
DNS traffic is becoming increasingly encrypted, making it difficult to see
domain name lookups and responses. Another reason is that the DNS names to
these services can change.}
\eprob

\prob{5}
What are the five header types in a network packet that make up a flow?

\answerbox{1}{Source and destination IP address, source and destination port,
protocol.}
\eprob

\prob{3}
List three advantages to active Internet measurement over passive Internet measurement.

\answerbox{1}{Direct measurement of desired effect, timing and frequency can
be controlled, little to no privacy risks. (There was a whole slide on this in
the board notes that we came up with in class, so anything from that slide, or
anything reasonable, will suffice.)}
\eprob


\section*{Feature Engineering}

\prob{3}
Features should be characteristic of fundamental differences between classes, rather than
simply characteristics of the dataset. Suppose you have a {\em single} packet
trace from the University of Chicago campus network, where Log4j scans are
being conducted at the same time as regular traffic.
You decide to use {\em only incoming network traffic} to train a detection model, using
features that include all of five the fields for incoming network
traffic, and the detection model works really well. But, when your friends at
Northwestern try
to use your model, it doesn't work at all. What feature or features might be at fault,
{\bf and why}?

\answerbox{0.75}{Destination IP address, because the destination IP addresses
will be different at Northwestern. (Other fields, like source port, may
be an acceptable answer if well-explained.)}
\eprob

\section*{Feedback}
\prob{1}
Interest (1=Boring!; 10=Amazing!):
\shortanswerbox{0.5}{5}
Difficulty (1=Too easy; 10=Too hard):
\shortanswerbox{0.5}{5}
\eprob
\prob{1}
1. One thing you like. 2. One suggestion for improvement:

\answerbox{0.75}{More free food.}
\eprob

